\chapter{实验设计与结果分析}
\label{chap:exp}

\section{实验目标与研究问题}
围绕提出的协同框架,本文设计以下研究问题:
\begin{enumerate}
    \item RQ1:相比主流序列/图/LLM基线,本文方法是否稳定提升推荐性能?
    \item RQ2:时间增强、双向转移、GCIM、PAM 等关键模块是否均有独立贡献?
    \item RQ3:在冷启动与跨城迁移场景中,方法是否仍具有鲁棒性?
\end{enumerate}

\section{数据集与评价指标}
\subsection{数据集}
与前期工作一致,实验采用Gowalla与Foursquare用于小模型验证;采用NYC、TKY、CA用于LLM路线与跨城泛化分析。各数据集均包含用户ID、POI ID、时间戳、经纬度与类别信息。

\subsection{评价指标}
采用准确率与排序质量指标:Acc@1、Acc@5、Acc@10、MRR(或MRR@5)。记第 $i$ 个样本中真实POI排名为 $rank_i$,则
\[
\text{Acc@k}=\frac{1}{N}\sum_{i=1}^{N}\mathbb{I}(rank_i\le k),\qquad
\text{MRR}=\frac{1}{N}\sum_{i=1}^{N}\frac{1}{rank_i}.
\]

\section{小模型实验:TSPM结果分析}
\subsection{总体对比结果}
表~\ref{tab:tspm-main} 给出了你前期小模型稿件中的核心结果。TSPM在Gowalla与Foursquare上均达到最优,说明“时间槽偏好 + 双向转移 + 动态图”对下一POI预测有稳定增益。

\begin{table}[htbp]
    \centering
    \caption{TSPM与基线在Gowalla/Foursquare上的结果(来自前期工作)}
    \label{tab:tspm-main}
    \resizebox{\textwidth}{!}{
    \begin{tabular}{l|cccc|cccc}
        \toprule
        \multirow{2}{*}{方法} & \multicolumn{4}{c|}{Gowalla} & \multicolumn{4}{c}{Foursquare} \\
        & Acc@1 & Acc@5 & Acc@10 & MRR & Acc@1 & Acc@5 & Acc@10 & MRR \\
        \midrule
        PRME & 0.0740 & 0.2146 & 0.2899 & 0.1503 & 0.0982 & 0.3167 & 0.4064 & 0.2040 \\
        STRNN & 0.0900 & 0.2120 & 0.2730 & 0.1508 & 0.2290 & 0.4310 & 0.5050 & 0.3248 \\
        DeepMove & 0.0625 & 0.1304 & 0.1594 & 0.0982 & 0.2400 & 0.4319 & 0.4742 & 0.3270 \\
        LBSN2Vec & 0.0864 & 0.1186 & 0.1390 & 0.1032 & 0.2190 & 0.3955 & 0.4621 & 0.2781 \\
        STGN & 0.0624 & 0.1586 & 0.2104 & 0.1125 & 0.2094 & 0.4734 & 0.5470 & 0.3283 \\
        LightGCN & 0.0428 & 0.1439 & 0.2115 & 0.1224 & 0.0540 & 0.1790 & 0.2710 & 0.1574 \\
        Flashback & 0.1158 & 0.2754 & 0.3479 & 0.1925 & 0.2496 & 0.5399 & 0.6236 & 0.3805 \\
        STAN & 0.0891 & 0.2096 & 0.2763 & 0.1523 & 0.2265 & 0.4515 & 0.5310 & 0.3420 \\
        GETNext & 0.1419 & 0.3270 & 0.4081 & 0.2294 & 0.2646 & 0.5640 & 0.6431 & 0.3988 \\
        Graph-Flashback & 0.1512 & 0.3425 & 0.4256 & 0.2422 & 0.2805 & 0.5757 & 0.6514 & 0.4136 \\
        \midrule
        TSPM & \textbf{0.1595} & \textbf{0.3520} & \textbf{0.4350} & \textbf{0.2509} & \textbf{0.2932} & \textbf{0.5978} & \textbf{0.6768} & \textbf{0.4301} \\
        \bottomrule
    \end{tabular}}
\end{table}

\subsection{消融实验结论}
在小模型稿件中,去除TSDG或去除双向转移建模(BTM)均会造成性能下降,验证了:
\begin{enumerate}
    \item 时间槽建模能够显著缓解“同一POI在不同时段行为模式不同”的问题;
    \item 转入/转出双向偏好比单向转移更能刻画真实出行路径。
\end{enumerate}

\section{大模型实验:GA-LLM结果分析}
\subsection{核心发现}
依据ICDE二轮稿与修订说明,GA-LLM通过GCIM和PAM在三个方面取得改进:
\begin{enumerate}
    \item \textbf{精度提升}:相对LLM4POI与多种基线,在Acc@1/Acc@5/MRR@5上取得明显优势(文中报告最高可达24.10\%相对提升);
    \item \textbf{空间幻觉缓解}:错误预测与真实POI的平均地理距离显著下降,说明GCIM有效约束了地理不合理生成;
    \item \textbf{跨城冷启动增强}:在训练城市与测试城市不一致时,GA-LLM仍优于文本LLM基线,显示更强可迁移性。
\end{enumerate}

\subsection{GCIM作用分析}
GCIM的贡献可概括为“离散层级结构 + 连续频域结构 + 地理一致性约束”三者协同:
\begin{enumerate}
    \item Quadkey层级编码保留区域-子区域多尺度信息;
    \item Fourier连续映射补足平滑空间变化;
    \item 测地约束将真实地理距离引入表示学习,减少语义空间扭曲。
\end{enumerate}

\subsection{PAM作用分析}
PAM将图模型POI表示映射到LLM语义空间,在目标POI未显式出现在输入历史时仍可通过转移先验进行推断。对比“仅token学习POI”的策略,PAM在复杂城市分布下更稳定。

\subsection{效率与可扩展性}
GA-LLM采用结构化坐标注入而非原始长文本坐标描述,减少了上下文冗余;同时LoRA微调降低了训练参数开销,具有较好的工程可部署性。

\section{融合模型的综合讨论}
将小模型与大模型能力融合后,模型具备如下互补机制:
\begin{enumerate}
    \item 小模型提供可解释时空结构先验(时间槽、转移图、双向偏好);
    \item 大模型提供强语义建模与跨场景泛化能力;
    \item 对齐模块将二者映射到统一语义空间,缓解结构信息在生成模型中的丢失。
\end{enumerate}

从误差类型看,融合模型可同时降低两类错误:
\begin{enumerate}
    \item 由文本语义偏置导致的“空间远跳错误”;
    \item 由局部历史过拟合导致的“重复历史POI错误”。
\end{enumerate}

\section{实验小结}
本章实验表明:
\begin{enumerate}
    \item 在小模型路线中,TSPM通过时间增强与双向转移显著提升基础预测能力;
    \item 在大模型路线中,GA-LLM通过GCIM/PAM显著提升空间一致性与冷启动鲁棒性;
    \item 二者融合后能够在精度、鲁棒性和可解释性之间取得更优平衡。
\end{enumerate}
