\chapter{大小模型协同学习方法}
\label{chap:method}

\section{总体框架与设计动机}
基于第\ref{chap:related}章的问题分析,本文方法由三个部分构成:
\begin{enumerate}
    \item 小模型分支 TSPM:学习时间敏感的时空转移结构;
    \item 大模型分支 GA-LLM:增强地理连续性建模与POI先验注入;
    \item 融合分支:通过嵌入对齐与两阶段训练实现协同优化。
\end{enumerate}

设计动机是将小模型的结构归纳偏置与大模型的语义推理能力进行互补融合,避免单一路线在精度、鲁棒性或泛化能力上的短板。

\section{小模型分支:TSPM}
\subsection{时间增强序列动态图(TSDG)}
为刻画不同时段的迁移差异,将一天划分为 $z$ 个时间槽 $\{T_1,\ldots,T_z\}$,并在各时间槽内构建POI转移子图。对当前POI嵌入 $\mathbf{e}_i$ 与时间槽嵌入 $\mathbf{t}_i$,定义时间感知的转出与转入表示:
\begin{equation}
\boldsymbol{\xi}^{out}_{i,T_i}=\sigma\left([\mathbf{e}_i\Vert \mathbf{t}_i]\mathbf{W}^{t}_{out}+\mathbf{b}^{t}_{out}\right),
\label{eq:c3-xi-out}
\end{equation}
\begin{equation}
\boldsymbol{\xi}^{in}_{j,T_i}=\sigma\left([\mathbf{e}_j\Vert \mathbf{t}_i]\mathbf{W}^{t}_{in}+\mathbf{b}^{t}_{in}\right).
\label{eq:c3-xi-in}
\end{equation}

\textbf{待验证命题:}显式时间分槽可提升模型对时段异质行为的建模能力(对应第\ref{chap:exp}章RQ2)。

\subsection{双向转移建模}
为同时建模“从哪里来”和“将去哪里”,采用双向对比损失:
\begin{equation}
\mathcal{L}_{time}=-\sum_t\log\sigma\left(\|\boldsymbol{\xi}^{out}_{i,T}-\boldsymbol{\xi}^{in}_{-,T}\|_2^2-\|\boldsymbol{\xi}^{out}_{i,T}-\boldsymbol{\xi}^{in}_{+,T}\|_2^2\right),
\label{eq:c3-ltime}
\end{equation}
其中 $+$ 与 $-$ 分别表示正负样本POI。

\textbf{待验证命题:}双向转移优于单向转移,可减少路径偏置并提升下一跳预测稳定性(对应RQ2)。

\subsection{序列偏好建模与动态图权重}
将最近 $k$ 个访问拼接得到序列表示:
\begin{equation}
\boldsymbol{\xi}_{seq}=\sigma\left([\mathbf{e}_t\Vert\mathbf{e}_{t-1}\Vert\cdots\Vert\mathbf{e}_{t-k}]\mathbf{W}^{s}+\mathbf{b}^{s}\right),
\label{eq:c3-xi-seq}
\end{equation}
并使用序列对比损失:
\begin{equation}
\mathcal{L}_{seq}=-\sum_t\log\sigma\left(\|\boldsymbol{\xi}_{seq}-\mathbf{e}^{-}_t\|_2^2-\|\boldsymbol{\xi}_{seq}-\mathbf{e}^{+}_{t+1}\|_2^2\right),
\label{eq:c3-lseq}
\end{equation}
综合得到
\begin{equation}
\mathcal{L}_{TSPM}=\alpha\mathcal{L}_{time}+\beta\mathcal{L}_{seq}.
\label{eq:c3-ltspm}
\end{equation}

据此定义动态图边权:
\begin{equation}
s^d_{i,j}=\exp\left(-\rho_1\|\boldsymbol{\xi}_{seq}-\mathbf{e}_j\|_2^2-\rho_2\|\boldsymbol{\xi}^{out}_{i,T}-\boldsymbol{\xi}^{in}_{j,T}\|_2^2\right).
\label{eq:c3-edgew}
\end{equation}

\textbf{待验证命题:}动态图权重可提升复杂迁移场景下的区分能力(对应RQ2、RQ4)。

\subsection{TiRNN预测头}
为建模多步历史依赖,TiRNN对过去 $K$ 个隐状态加权融合:
\begin{equation}
\mathbf{c}_t=\sum_{k=1}^{K}\alpha_k(\mathbf{h}_{t-k}\circ\mathbf{r}_k),
\label{eq:c3-context}
\end{equation}
\begin{equation}
\mathbf{h}_t=\sigma(\kappa\mathbf{v}_t+\mathbf{c}_t),\qquad
\hat{\mathbf{y}}_t=\text{Softmax}(\mathbf{W}_f[\hat{\mathbf{h}}_t\Vert\mathbf{E}_u]).
\label{eq:c3-tirnn}
\end{equation}

\section{大模型分支:GA-LLM}
\subsection{GCIM:地理坐标注入模块}
GCIM采用“层级离散 + 连续频域”双分支编码:
\begin{equation}
\mathbf{E}_{fourier}=\frac{1}{\sqrt{M}}\left[\cos(\mathbf{g}\mathbf{W}_s^\top)\,\Vert\,\sin(\mathbf{g}\mathbf{W}_s^\top)\right],
\label{eq:c3-fourier}
\end{equation}
\begin{equation}
\mathbf{E}_{GPS}=\mathbf{W}_{GPS}[\mathbf{E}_{quad}\Vert\mathbf{E}_{fourier}].
\label{eq:c3-egps}
\end{equation}

该模块通过测地一致性约束降低空间幻觉。

\textbf{待验证命题:}GCIM可显著改善地理一致性并降低远跳错误(对应RQ3、RQ4)。

\subsection{PAM:POI对齐模块}
PAM将图模型POI表示映射到LLM语义空间:
\begin{equation}
\mathbf{E}_{poi}=\text{PAM}(\mathbf{e}_{poi})=\mathbf{W}_{p}\mathbf{e}_{poi}+\mathbf{b}_{p}.
\label{eq:c3-epoi}
\end{equation}

相比纯token方式,PAM可显式注入转移先验。

\textbf{待验证命题:}PAM可提升目标POI未显式出现时的预测能力(对应RQ3、RQ4)。

\subsection{结构化提示构造}
采用“轨迹文本 + 专用token”混合提示:
\begin{quote}
This is the historical trajectory of user u: ... <POI $p_i$>, <GPS $g_i$> ... Which POI will user u visit next?
\end{quote}
其中 `<GPS>` 由GCIM编码,`<POI>` 由PAM编码。

\section{融合策略:协同训练与推理}
\subsection{两阶段训练流程}
阶段一冻结LLM主体,仅训练GCIM/PAM与映射层,建立稳定对齐;阶段二采用LoRA微调注意力层,联合优化序列与生成目标:
\begin{equation}
\mathcal{L}_{total}=\lambda_1\mathcal{L}_{gen}+\lambda_2\mathcal{L}_{geo}+\lambda_3\mathcal{L}_{align}+\lambda_4\mathcal{L}_{TSPM}.
\label{eq:c3-ltotal}
\end{equation}

\subsection{推理机制}
推理时,先由GCIM与PAM将地理与转移信息注入提示,再由LLM输出候选分布;可选地结合TSPM分数进行重排序,以得到最终Top-$K$结果。

\section{复杂度与可扩展性讨论}
训练成本主要来自三部分:TSPM动态图更新、LLM前向计算与跨模型对齐。相较于全参数微调,LoRA将可训练参数控制在低秩子空间,显著降低显存与训练时间开销。推理阶段可按场景启用“仅LLM预测”或“LLM+TSPM重排序”两种模式,在效果与时延之间灵活折中。

\section{本章小结}
本章在统一框架下给出了小模型分支、大模型分支与协同训练机制的完整设计,并明确了各模块在实验章中对应的验证命题。下一章将围绕研究问题给出可复现实验设置与系统评估结果。
