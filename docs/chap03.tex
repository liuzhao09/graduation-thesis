\chapter{大小模型协同学习方法}
\label{chap:method}

\section{任务定义与总体框架}
设用户集合为 $\mathcal{U}$,POI集合为 $\mathcal{L}$。用户 $u\in\mathcal{U}$ 的签到序列记为
\[
\mathcal{T}_u=\{(p_1,t_1,g_1,c_1),\ldots,(p_n,t_n,g_n,c_n)\},
\]
其中 $p_i$ 为POI ID,$t_i$ 为时间戳,$g_i=(lat_i,lon_i)$ 为经纬度,$c_i$ 为类别。目标是学习函数
\[
\hat{p}_{n+1}=f(\mathcal{T}_u),\quad \hat{p}_{n+1}\in\mathcal{L},
\]
使真实下一POI在候选排序中尽可能靠前。

本文框架由三部分组成:
\begin{enumerate}
    \item 小模型分支:TSPM(Time-enhanced Sequential Prediction Model);
    \item 大模型分支:GA-LLM(Geography-Aware LLM);
    \item 融合分支:嵌入对齐与两阶段协同训练。
\end{enumerate}

\section{小模型分支:TSPM}
\subsection{时间增强序列动态图(TSDG)}
传统连续序列默认不同时间段共享同一转移机制,难以刻画“早高峰通勤”与“夜间娱乐”行为差异。为此,将一天划分为 $z$ 个时间槽 $\{T_1,\ldots,T_z\}$,在每个时间槽内构建POI转移子图。对当前POI嵌入 $\mathbf{e}_i$ 与时间槽嵌入 $\mathbf{t}_i$,定义时间感知的转出向量:
\[
\boldsymbol{\xi}^{out}_{i,T_i}=\sigma\left([\mathbf{e}_i\Vert \mathbf{t}_i]\mathbf{W}^{t}_{out}+\mathbf{b}^{t}_{out}\right),
\]
对应地定义转入向量:
\[
\boldsymbol{\xi}^{in}_{j,T_i}=\sigma\left([\mathbf{e}_j\Vert \mathbf{t}_i]\mathbf{W}^{t}_{in}+\mathbf{b}^{t}_{in}\right).
\]

\subsection{双向转移建模}
为同时建模“从哪里来”和“将去哪里”,采用双向对比损失:
\[
\mathcal{L}_{time}=-\sum_t\log\sigma\left(\|\boldsymbol{\xi}^{out}_{i,T}-\boldsymbol{\xi}^{in}_{-,T}\|_2^2-\|\boldsymbol{\xi}^{out}_{i,T}-\boldsymbol{\xi}^{in}_{+,T}\|_2^2\right),
\]
其中 $+$ 与 $-$ 分别表示正负样本POI。该损失鼓励当前POI在给定时间槽下更接近真实下一跳的转入表示。

\subsection{序列偏好建模与动态图权重}
将最近 $k$ 个访问拼接后得到序列表示:
\[
\boldsymbol{\xi}_{seq}=\sigma\left([\mathbf{e}_t\Vert\mathbf{e}_{t-1}\Vert\cdots\Vert\mathbf{e}_{t-k}]\mathbf{W}^{s}+\mathbf{b}^{s}\right).
\]
对应序列对比损失为:
\[
\mathcal{L}_{seq}=-\sum_t\log\sigma\left(\|\boldsymbol{\xi}_{seq}-\mathbf{e}^{-}_t\|_2^2-\|\boldsymbol{\xi}_{seq}-\mathbf{e}^{+}_{t+1}\|_2^2\right).
\]
综合损失:
\[
\mathcal{L}_{TSPM}=\alpha\mathcal{L}_{time}+\beta\mathcal{L}_{seq}.
\]

学习完成后,定义动态边权:
\[
s^d_{i,j}=\exp\left(-\rho_1\|\boldsymbol{\xi}_{seq}-\mathbf{e}_j\|_2^2-\rho_2\|\boldsymbol{\xi}^{out}_{i,T}-\boldsymbol{\xi}^{in}_{j,T}\|_2^2\right),
\]
并据此构建时间增强动态图进行邻域聚合。

\subsection{TiRNN预测头}
为显式建模多步历史影响,TiRNN将过去 $K$ 个隐状态加权融合:
\[
\mathbf{c}_t=\sum_{k=1}^{K}\alpha_k(\mathbf{h}_{t-k}\circ\mathbf{r}_k),
\]
\[
\mathbf{h}_t=\sigma(\kappa\mathbf{v}_t+\mathbf{c}_t),\qquad
\hat{\mathbf{y}}_t=\text{Softmax}(\mathbf{W}_f[\hat{\mathbf{h}}_t\Vert\mathbf{E}_u]).
\]
训练目标采用交叉熵与$L_2$正则。

\section{大模型分支:GA-LLM}
\subsection{问题动机}
基于LLM的POI推荐虽具语义优势,但存在“空间连续性缺失”和“转移先验不足”两类核心问题:
\begin{enumerate}
    \item 经纬度离散分词后,近邻点可能映射到语义距离较远的token空间;
    \item 仅依赖文本提示时,模型倾向重复历史出现POI,难以推断未显式出现但转移概率高的目标POI。
\end{enumerate}

\subsection{GCIM:地理坐标注入模块}
GCIM由双分支组成:
\begin{enumerate}
    \item 层级离散分支(Quadkey n-gram):编码多尺度地理网格结构;
    \item 连续频域分支(Fourier):编码平滑地理距离变化。
\end{enumerate}

连续频域分支写作:
\[
\mathbf{E}_{fourier}=\frac{1}{\sqrt{M}}\left[\cos(\mathbf{g}\mathbf{W}_s^\top)\,\Vert\,\sin(\mathbf{g}\mathbf{W}_s^\top)\right],
\]
其中 $\mathbf{g}=(lat,lon)$。双分支融合后通过投影映射至LLM语义空间:
\[
\mathbf{E}_{GPS}=\mathbf{W}_{GPS}[\mathbf{E}_{quad}\Vert\mathbf{E}_{fourier}].
\]

为保持语义空间中的地理一致性,引入测地一致性损失,约束嵌入距离与真实地理距离单调一致,从而降低空间幻觉预测。

\subsection{PAM:POI对齐模块}
PAM将图模型产生的低维POI向量 $\mathbf{e}_{poi}\in\mathbb{R}^{d}$ 投影至LLM空间:
\[
\mathbf{E}_{poi}=\text{PAM}(\mathbf{e}_{poi})=\mathbf{W}_{p}\mathbf{e}_{poi}+\mathbf{b}_{p}.
\]

与仅将POI当作独立token的策略不同,PAM显式引入跨POI转移先验,使模型能在目标POI未出现在输入历史时仍基于结构化关系完成预测。

\subsection{结构化提示构造}
采用“轨迹文本 + 专用token”混合提示:
\begin{quote}
This is the historical trajectory of user u: ... <POI $p_i$>, <GPS $g_i$> ... Which POI will user u visit next?
\end{quote}
其中 `<GPS>` 由GCIM编码,`<POI>` 由PAM编码。该策略在不显著增加上下文长度的前提下注入关键地理与转移信息。

\section{融合策略:大小模型协同训练}
\subsection{两阶段训练流程}
\textbf{阶段一(对齐阶段)}:冻结LLM主体,仅训练GCIM/PAM与跨模型映射层,获得稳定空间与POI对齐表征。  
\textbf{阶段二(协同阶段)}:采用LoRA微调LLM注意力层,联合优化序列目标与生成目标。

总目标可写为:
\[
\mathcal{L}_{total}=\lambda_1\mathcal{L}_{gen}+\lambda_2\mathcal{L}_{geo}+\lambda_3\mathcal{L}_{align}+\lambda_4\mathcal{L}_{TSPM}.
\]
其中 $\mathcal{L}_{gen}$ 为自回归生成损失,$\mathcal{L}_{geo}$ 为地理一致性损失,$\mathcal{L}_{align}$ 为POI语义对齐损失。

\subsection{推理机制}
推理时,将用户近期轨迹编码为结构化提示,GCIM与PAM分别注入地理与转移信息,LLM输出候选POI分布;同时可利用小模型输出进行重排序,得到最终Top-$K$推荐列表。

\section{本章小结}
本章构建了从“小模型时空建模”到“大模型语义增强”再到“统一协同训练”的完整方法体系。该体系针对POI推荐中的时间异质性、空间连续性与转移先验三类核心难题给出统一解决方案,为下一章实验验证奠定基础。
