\chapter*{后记}
\addcontentsline{toc}{chapter}{后记}
\markboth{后记}{}

三年的研究生学习即将结束,从入学时对“做科研”只有模糊认识,到今天能够较为独立地完成问题定义、方法设计、实验验证与论文写作,这一过程远比我最初想象的更漫长,也更珍贵。回顾这段经历,自己真正收获的不只是论文结果本身,更是面对复杂问题时的耐心、面对失败实验时的韧性,以及在反复修改中逐步接近严谨表达的能力。

读研初期,我花了大量时间补齐推荐系统、时空建模和大模型相关基础。很多看似“懂了”的内容,在真正动手复现和推导时又会暴露理解盲区。进入课题后,围绕 Next POI 推荐这一方向,我经历了从传统序列方法到图建模,再到大语言模型增强方案的不断尝试。期间有过实验长期不收敛、结果波动难以解释、写作结构反复推翻重来的阶段,也正是在这些具体而琐碎的困难里,我逐渐理解了研究工作的核心不是追求“看起来漂亮”的结果,而是对问题边界、方法假设和证据链条保持诚实与清醒。

本文最终形成的两条主线工作(TSPM 与 GA-LLM)并非一蹴而就,而是在多轮讨论、验证和修正中逐步沉淀出来的。前者让我更深刻地认识到时空行为中的结构规律与时间异质性;后者让我意识到大模型能力的边界,以及结构先验注入在实际任务中的必要性。无论结果大小,这些探索过程都成为我研究训练中最重要的部分。

衷心感谢导师在选题把关、方法路线、实验设计和论文写作上的持续指导。导师严谨务实的科研态度、对细节的高标准要求和对学生成长的耐心投入,使我在每一个关键节点都能及时校正方向、稳步推进。导师不仅教会我“怎么做研究”,也让我理解了“为什么要这样做研究”。这份影响将长期伴随我今后的学习与工作。

感谢课题组各位老师和同学在数据处理、代码复核、论文讨论中的帮助。每一次组会交流和问题争论,都在推动我把模糊想法变成可验证命题。感谢学院与学校提供的学习平台与研究条件,感谢家人和朋友在读研阶段给予的理解、支持与鼓励,使我能够专注地走完这段旅程。

谨以此文,向所有在研究生阶段给予我指导、帮助和陪伴的人致以最诚挚的感谢。

% 后记结束后恢复普通页样式
\pagestyle{thesisnormal}
