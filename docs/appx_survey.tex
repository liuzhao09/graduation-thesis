\chapter{扩展文献综述与方法对比}
\label{appx:survey}

本附录在第\ref{chap:related}章基础上进行进一步综述,目的不是重复结论,而是补充“问题演化脉络、方法选择依据与研究决策过程”,以展示论文工作中完整的文献调研与方案筛选过程。

\section{问题演化脉络}
\subsection{从静态推荐到序列推荐}
早期推荐系统主要关注用户长期偏好建模,典型假设是偏好在短期内相对稳定。随着移动场景兴起,研究逐步转向“下一步行为预测”,即从静态偏好估计过渡到动态决策建模。该演化意味着模型需要同时处理时间顺序、上下文变化和短期意图漂移。

\subsection{从序列建模到结构建模}
仅依赖序列模型虽能捕捉局部转移,但对高阶关系与稀疏监督适应有限。图建模通过全局关系传播缓解了这一问题,特别是在长尾POI和冷用户场景下表现更稳健。与此同时,图方法也引入了动态更新与计算复杂度挑战。

\subsection{从结构建模到语义增强}
大语言模型的出现使推荐研究进入“语义增强”阶段。LLM能够处理复杂文本上下文并具备一定泛化能力,但在Next POI这类强地理约束任务中,若缺乏结构先验,易出现语义合理但地理不合理的预测。由此催生“结构-语义协同”的研究方向。

\section{方法路线扩展对比}
\subsection{序列路线扩展分析}
序列路线的核心在于历史依赖建模能力。其优势是训练效率较高、实现成熟、工程部署路径清晰;不足在于高阶迁移表达弱、跨域泛化有限。研究实践表明,单纯增加层数或注意力头数并不能系统解决地理一致性问题。

\subsection{图路线扩展分析}
图路线擅长利用POI之间的关系结构,在稀疏数据与长尾分布中优势明显。其主要挑战包括动态图维护成本、噪声边传播以及与语义模型对齐难度。若缺乏合适的对齐机制,图信息很难在LLM空间中被有效利用。

\subsection{LLM路线扩展分析}
LLM路线在开放语义理解和上下文推理方面具有天然优势,尤其适合处理冷启动文本描述和复杂意图表达。但LLM并不天然满足空间连续性,也不具备显式迁移图先验。针对Next POI任务,必须引入专门地理与结构注入策略。

\section{本文方案选择过程}
\subsection{候选方案A:纯结构增强}
该方案以图与序列模块为主,不引入LLM。优点是实现成本低、训练稳定;缺点是语义泛化上限受限,难覆盖复杂意图表达场景。

\subsection{候选方案B:纯语义增强}
该方案以LLM为主,仅使用提示工程注入历史。优点是开发迭代快;缺点是地理一致性风险高,且对结构关系依赖隐式学习,稳定性不足。

\subsection{候选方案C:分阶段协同}
该方案即本文采用路线:先对齐后协同,先解决“可读性”,再优化“可用性”。在综合精度、稳定性、可解释性和部署成本后,该方案成为最终选择。

\section{设计决策清单}
为体现研究过程中的取舍逻辑,本文总结关键决策如下:
\begin{enumerate}
    \item 是否全参微调LLM:否,采用LoRA以控制训练成本;
    \item 是否仅使用原始坐标token:否,引入GCIM保障空间一致性;
    \item 是否仅依赖语义匹配:否,引入PAM注入结构先验;
    \item 是否端到端一次性联合训练:否,采用两阶段减少梯度冲突;
    \item 是否只报告主表结果:否,增加消融、诊断与效率分析。
\end{enumerate}

\section{模型比较维度扩展}
\subsection{精度维度}
关注Acc@K、MRR、NDCG等指标,评价模型在命中率与排序质量上的综合表现。

\subsection{稳定性维度}
关注不同随机种子、不同数据划分、不同参数设置下的结果波动,强调结论可重复性。

\subsection{可解释性维度}
关注模型能否解释“为何推荐该POI”,尤其是地理可达性与转移路径合理性。

\subsection{部署性维度}
关注训练资源、推理延迟、参数规模、在线更新难度,强调研究结果可落地。

\section{对现有工作的再评价}
在扩展调研中,本文形成以下再评价观点:
\begin{enumerate}
    \item 单路线方法在特定指标上可能优异,但在真实系统的多目标约束下通常存在短板;
    \item 纯语义方法若缺乏结构约束,在地理场景中容易产生“看似合理”的错误;
    \item 纯结构方法若缺乏语义能力,在复杂意图和长尾场景中泛化受限;
    \item 协同路线的关键不在“拼接”,而在“对齐、分工与联合优化机制”。
\end{enumerate}

\section{研究边界与开放问题}
尽管本文框架验证了协同路线有效性,仍存在若干开放问题:
\begin{enumerate}
    \item 如何在在线反馈闭环中动态更新对齐模块而不破坏稳定性;
    \item 如何将更多外部上下文融入协同框架并保持计算可控;
    \item 如何在跨文化与跨城市差异显著场景中建立统一迁移机制;
    \item 如何构建更贴近线上收益的评价体系并与离线指标对齐。
\end{enumerate}

\section{附录小结}
本附录补充了文献调研深度与方法取舍过程,目的在于说明本文并非从单一模型出发“事后解释”,而是在系统评估多种路线后形成协同方案。这一过程体现了研究设计的完整性与论文工作的严谨态度。
