\newcommand{\enline}[3]{\noindent\hspace*{#1}#2\leaders\hbox to .45em{\hss.\hss}\hfill #3\par}

\enline{0em}{\textbf{Abstract (In Chinese)}}{I}
\enline{0em}{\textbf{Abstract (In English)}}{II}
\enline{0em}{\textbf{List of Abbreviations}}{IX}

\enline{0em}{\textbf{Chapter 1: Introduction}}{1}
\enline{1.8em}{Section 1.1: Chapter Introduction}{1}
\enline{1.8em}{Section 1.2: Research Background and Significance}{1}
\enline{3.6em}{Subsection 1.2.1: Research Background}{1}
\enline{3.6em}{Subsection 1.2.2: Research Significance}{2}
\enline{1.8em}{Section 1.3: Research Questions, Problem Statement, and Core Challenges}{2}
\enline{3.6em}{Subsection 1.3.1: Research Question Decomposition}{2}
\enline{3.6em}{Subsection 1.3.2: Problem Statement}{3}
\enline{3.6em}{Subsection 1.3.3: Core Challenges}{3}
\enline{1.8em}{Section 1.4: Research Idea, Work Plan, and Literature Positioning}{4}
\enline{3.6em}{Subsection 1.4.1: Research Idea and Technical Route}{4}
\enline{3.6em}{Subsection 1.4.2: Research Content and Work Plan}{5}
\enline{3.6em}{Subsection 1.4.3: Literature Context and Positioning}{5}
\enline{1.8em}{Section 1.5: Main Contributions and Thesis Organization}{6}
\enline{3.6em}{Subsection 1.5.1: Main Contributions}{6}
\enline{3.6em}{Subsection 1.5.2: Thesis Organization}{6}
\enline{1.8em}{Section 1.6: Chapter Summary}{6}

\enline{0em}{\textbf{Chapter 2: Related Work and Problem Analysis}}{7}
\enline{1.8em}{Section 2.1: Chapter Introduction}{7}
\enline{1.8em}{Section 2.2: Related Work Survey and Research Progress}{7}
\enline{3.6em}{Subsection 2.2.1: Small-model Route: Sequence and Graph Methods}{7}
\enline{3.6em}{Subsection 2.2.2: Large-model Route: LLM-driven Recommendation}{8}
\enline{3.6em}{Subsection 2.2.3: Generative Recommendation and Explicit Reasoning}{9}
\enline{3.6em}{Subsection 2.2.4: Chinese Research Context and Local Insights}{11}
\enline{3.6em}{Subsection 2.2.5: Extended Literature Synthesis}{11}
\enline{1.8em}{Section 2.3: Task Definition and Notation}{12}
\enline{3.6em}{Subsection 2.3.1: Task Scope and Modeling Assumptions}{12}
\enline{3.6em}{Subsection 2.3.2: Metrics and Experimental Protocol}{13}
\enline{1.8em}{Section 2.4: Research Gaps and Modeling Principles}{14}
\enline{3.6em}{Subsection 2.4.1: Mapping Research Gaps to RQs}{15}
\enline{1.8em}{Section 2.5: Chapter Summary}{15}

\enline{0em}{\textbf{Chapter 3: Collaborative Learning Method of Large and Small Models}}{16}
\enline{1.8em}{Section 3.1: Chapter Introduction}{16}
\enline{1.8em}{Section 3.2: Overall Framework and Design Motivation}{16}
\enline{3.6em}{Subsection 3.2.1: Input Construction and Data Flow Definition}{17}
\enline{3.6em}{Subsection 3.2.2: Core Symbols and RQ Mapping}{17}
\enline{3.6em}{Subsection 3.2.3: Collaborative Design Principles}{17}
\enline{1.8em}{Section 3.3: Key Module Design}{18}
\enline{3.6em}{Subsection 3.3.1: Small-model Branch: TSPM}{18}
\enline{3.6em}{Subsection 3.3.2: Large-model Branch: GA-LLM}{23}
\enline{1.8em}{Section 3.4: Alignment and Collaborative Training Strategy}{29}
\enline{3.6em}{Subsection 3.4.1: Two-stage Training Workflow}{29}
\enline{3.6em}{Subsection 3.4.2: Coupling Path and Parameter Update Mechanism}{30}
\enline{3.6em}{Subsection 3.4.3: Training Workflow (Implementation Perspective)}{31}
\enline{3.6em}{Subsection 3.4.4: Inference Mechanism}{31}
\enline{3.6em}{Subsection 3.4.5: Parallel Branches and Result Comparison}{31}
\enline{3.6em}{Subsection 3.4.6: Key Hyperparameters and Default Settings}{31}
\enline{3.6em}{Subsection 3.4.7: Complexity and Scalability Discussion}{31}
\enline{3.6em}{Subsection 3.4.8: Engineering Details and Deployment Strategy}{33}
\enline{1.8em}{Section 3.5: Method Comparison and Discussion}{34}
\enline{1.8em}{Section 3.6: Chapter Summary}{35}

\enline{0em}{\textbf{Chapter 4: Experimental Design and Result Analysis}}{36}
\enline{1.8em}{Section 4.1: Chapter Introduction}{36}
\enline{1.8em}{Section 4.2: Experimental Objectives and Research Questions}{36}
\enline{1.8em}{Section 4.3: Experimental Setup}{36}
\enline{3.6em}{Subsection 4.3.1: Datasets and Preprocessing}{36}
\enline{3.6em}{Subsection 4.3.2: Metrics and Evaluation Protocol}{38}
\enline{3.6em}{Subsection 4.3.3: Baselines and Grouping}{38}
\enline{3.6em}{Subsection 4.3.4: Implementation Details and Statistical Tests}{39}
\enline{1.8em}{Section 4.4: Main Results: Dual-route Models vs Baselines (RQ1)}{39}
\enline{3.6em}{Subsection 4.4.1: Main Results of Small-model Route}{39}
\enline{3.6em}{Subsection 4.4.2: Main Results of Large-model Route}{39}
\enline{3.6em}{Subsection 4.4.3: Representative Difference Sampling Analysis}{41}
\enline{3.6em}{Subsection 4.4.4: Layered Interpretation of Results}{41}
\enline{1.8em}{Section 4.5: Mechanism Verification and Diagnostic Analysis (RQ2--RQ4)}{41}
\enline{3.6em}{Subsection 4.5.1: Ablation and Diagnostics of Small-model Branch (RQ2)}{41}
\enline{3.6em}{Subsection 4.5.2: Ablation and Diagnostics of Large-model Branch (RQ3)}{42}
\enline{3.6em}{Subsection 4.5.3: Complementarity Analysis of Dual Routes (RQ4)}{47}
\enline{1.8em}{Section 4.6: Efficiency and Scalability Analysis (RQ5)}{48}
\enline{3.6em}{Subsection 4.6.1: Model Scale and Training Strategy Experiments}{49}
\enline{3.6em}{Subsection 4.6.2: Deployment Strategy}{49}
\enline{1.8em}{Section 4.7: Chapter Summary}{50}

\enline{0em}{\textbf{Conclusion}}{51}
\enline{0em}{\textbf{References}}{53}
\enline{0em}{\textbf{Appendix}}{60}
\enline{0em}{\textbf{Postscript}}{61}
