\chapter{绪论}
\label{chap:intro}

\section{本章引言}
本章作为全文的总体导引,主要回答“为什么做、做什么、如何做、做出了什么”四个核心问题。首先从推荐系统向线下时空决策延展的现实背景出发,说明Next POI任务的研究价值;随后给出本文关注的问题边界与核心挑战,明确现有方法在时间异质性、空间一致性和转移先验注入方面的不足;在此基础上概述本文的大小模型协同技术路线、主要创新点与章节组织关系,为后续相关研究、方法细节和实验验证建立统一叙事主线。

\section{研究背景与研究意义}
\subsection{研究背景}
在数字经济快速发展的背景下,推荐系统已成为互联网平台的核心基础能力,并持续重塑用户的信息获取与消费决策方式\cite{koren2009matrix,rendle2009bpr,he2017ncf,zhang2023sequentialsurvey,liu2024cn1}。在电商场景中,平台通过个性化推荐连接“人-货-场”,显著提升商品发现效率与交易转化;在内容分发场景中,推荐机制持续影响用户的注意力分配与内容消费习惯,这一趋势在工业界与学术界均有系统讨论\cite{chen2024rsfm,gao2023llm4rs,li2024cn4,chen2024cn9,gao2024cn10}。可以看到,推荐系统已从单一功能模块演进为平台竞争力的关键基础设施。

随着推荐范式由“线上内容匹配”扩展到“线下服务决策支持”,位置感知推荐的重要性持续上升\cite{gao2022spatiotemporalsurvey,zhao2022cn5,he2021cn6,liu2023cn13,zheng2024cn14}。在本地生活与出行场景中,系统不仅要回答“用户喜欢什么”,还需要在时空约束下回答“用户此刻去哪里、下一步可能前往何处”,这使地理位置从辅助特征转化为核心变量\cite{cheng2013where,he2016inferring,GeoSAN,STiSAN,li2020gcnpoi}。因此,下一兴趣点推荐(Next Point-of-Interest Recommendation, Next POI)成为连接推荐系统、地理信息建模与城市计算的重要问题\cite{sun2022cn7,yang2023cn8,Pasquale2024City,M3PA,liu2024mobilityfm}。

与传统项目推荐相比,Next POI任务受地理可达性、时间节律、活动语义与行为路径连续性的共同约束\cite{PRME,ST-RNN,HST-LSTM,STAN,GETNext}。随着移动终端与位置服务平台普及,签到轨迹数据不仅记录“去过哪里”,还隐含“何时去、从哪里来、下一步去哪”的动态规律,为时空序列建模提供了基础\cite{feng2018deepmove,yang2020flashback,GraphFlashback,SNPM,STHGCN}。这一数据与问题特性也使其在学术研究与工程应用层面都具有持续价值\cite{AGCL,MTNet,ROTAN,TSPN-RA,DCPR}。

\subsection{研究意义}
本文工作的意义主要体现在理论与应用两个层面。
\begin{enumerate}
    \item \textbf{理论意义}:Next POI任务位于“时空序列建模 + 图结构学习 + 语言语义推理”的交叉区域\cite{zhang2023sequentialsurvey,wu2022gnnrecsurvey,cao2024trajllm,li2024spatialfm,gao2023llm4rs}。围绕该任务构建统一框架,有助于回答结构归纳偏置与大模型语义能力如何协同的问题。
    \item \textbf{方法意义}:现有方法多在单一维度较优,例如序列方法擅长局部时序\cite{hidasi2016gru4rec,li2017narm,liu2018stamp,hidasi2018gru4rec2,qin2020fmlp}、图方法擅长高阶关系\cite{ying2018pin,wu2019session,he2020lightgcn,wang2019kgat,lin2022simgcl}、LLM方法擅长语义泛化\cite{LLMRec,CoLLM,CoRAL,LLaRA,GraphGPT}。本文尝试给出可部署的融合路径。
    \item \textbf{应用意义}:在本地生活、城市出行和文旅推荐等场景中,系统需在有限时空约束下提供可执行推荐\cite{gao2022spatiotemporalsurvey,liu2023cn13,zheng2024cn14,wang2024georecllm,zheng2024poillm}。若预测结果与真实地理规律不一致,即使语义上合理也难形成有效服务。
\end{enumerate}

\section{研究问题、问题陈述与核心挑战}
\subsection{研究问题分解}
为避免“问题过大、方法过散”,本文将总体目标分解为五个可验证子问题,并在后文通过RQ体系进行闭环验证:
\begin{enumerate}
    \item \textbf{P1:时段异质性建模}。同一POI在不同时间段具有不同的转移规律,如何以低额外成本显式建模该异质性;
    \item \textbf{P2:双向转移表达}。下一跳预测不仅取决于“当前点指向谁”,也取决于“目标点通常从哪里来”,如何同时表达转出与转入偏好;
    \item \textbf{P3:地理连续性注入}。LLM语义空间不天然满足地理邻近关系,如何将经纬度与空间层级先验注入语义编码流程;
    \item \textbf{P4:结构先验跨空间对齐}。图模型学得的POI关系如何映射到LLM语义空间,并在推理阶段持续发挥作用;
    \item \textbf{P5:协同训练可部署}。在效果提升之外,如何保证训练成本、推理时延与参数规模仍处于可接受范围。
\end{enumerate}
通过上述分解,本文将“是否有效”与“为什么有效”区分处理,从而避免仅凭主结果表格给出结论。

\subsection{问题陈述}
本文关注的核心问题是:给定用户历史签到轨迹及其时间、空间和语义上下文,预测用户下一时刻最可能访问的POI,并输出Top-$K$候选列表。该问题兼具序列预测与结构推断属性,要求模型同时具备局部时空建模能力与跨场景泛化能力。

本文不在绪论中展开完整符号体系与公式定义,统一的任务定义与符号约定放在第\ref{chap:related}章给出,作为全文的单一依据。

\subsection{核心挑战}
尽管现有研究已在时空推荐方向取得进展,面向真实出行场景的Next POI建模仍面临以下挑战:
\begin{enumerate}
    \item \textbf{时间异质性挑战}:同一POI在不同时段的转移模式差异显著,统一转移机制容易造成偏差\cite{ST-RNN,zhao2019stgn,STAN,STiSAN,MTNet};
    \item \textbf{空间连续性挑战}:经纬度与语义表示空间之间缺乏天然同构,纯文本建模容易产生地理不一致预测\cite{GeoSAN,li2020gcnpoi,li2024large,LLM4POI,Jonathan2023Gpt};
    \item \textbf{转移先验注入挑战}:仅依赖文本上下文时,模型难以充分利用POI图中的高阶转移关系\cite{GraphFlashback,SNPM,STHGCN,SeCor,STKG-PLM};
    \item \textbf{协同优化挑战}:小模型结构先验与大模型语义能力在统一框架内稳定协同仍缺乏系统方案\cite{ReLLa,CoRAL,xu2024hybridrec,wu2024ragrec,qu2024onerecthink}。
\end{enumerate}

\begin{figure}[htbp]
    \centering
    \includegraphics[width=0.92\textwidth]{fig1_GCIM_c1.pdf}
    \caption{空间连续性挑战示意:原始坐标文本难以稳定保持地理邻近关系,导致语义空间与地理空间错位。}
    \label{fig:c1-gcim-motivation}
\end{figure}

如图\ref{fig:c1-gcim-motivation}所示,直接将坐标作为文本输入时,模型难以在语义空间中保持真实地理邻近结构。该现象说明Next POI任务需要专门的地理编码机制,而非仅依赖通用文本建模。基于这一挑战,本文在第\ref{chap:method}章设计GCIM模块以增强空间一致性。

\begin{figure}[htbp]
    \centering
    \includegraphics[width=0.92\textwidth]{fig2_PAM_c2.pdf}
    \caption{转移先验注入挑战示意:仅依赖token语义难以恢复POI图中的高阶迁移关系。}
    \label{fig:c1-pam-motivation}
\end{figure}

如图\ref{fig:c1-pam-motivation}所示,若缺少结构化对齐路径,模型对“未显式出现但高转移概率”的候选点识别能力不足。该现象说明POI关系先验需要显式注入语义空间,而非完全依赖隐式学习。基于这一挑战,本文在第\ref{chap:method}章设计PAM模块以提升转移感知能力。

\section{研究思路、工作安排与文献定位}
\subsection{研究思路与技术路线}
针对上述挑战,本文采用“大小模型协同学习”的总体路线:
\begin{enumerate}
    \item 在小模型侧构建时间增强序列动态图与双向转移机制,学习稳定时空结构先验\cite{yin2023next,yang2022getnext,rao2022graphflashback,AGCL,FHCRec};
    \item 在大模型侧设计地理坐标注入模块与POI对齐模块,增强LLM的空间一致性与转移感知能力\cite{li2024large,LLM4POI,wang2024georecllm,zheng2024poillm,cao2024trajllm};
    \item 通过嵌入对齐与两阶段训练,将结构知识与语义推理能力融合到统一框架中\cite{LLMRec,CoLLM,LLaRA,qu2024onerec,wu2024thinkrec}。
\end{enumerate}

该路线的目标不是简单叠加模型,而是在可部署约束下实现“结构归纳偏置 + 语义泛化能力”的互补增益。

\subsection{研究内容与工作安排}
围绕上述路线,本文研究内容可进一步细化为四项工作:
\begin{enumerate}
    \item \textbf{任务分析与基线整理}:系统梳理Next POI领域中的序列、图与LLM三类方法,统一问题定义、指标解释与实验比较边界;
    \item \textbf{小模型机制设计}:围绕时间异质性与迁移方向性,设计TSDG与双向转移建模,并通过动态图权重增强复杂场景区分能力;
    \item \textbf{大模型增强设计}:围绕地理连续性与转移先验缺失,设计GCIM与PAM模块,将结构先验以可训练方式注入LLM;
    \item \textbf{实验验证与诊断分析}:基于RQ组织实验证据,覆盖主结果、消融、误差类型、效率开销与部署讨论,形成“设置-结果-解释”闭环。
\end{enumerate}
从论文完成度角度,本文不仅给出模型结果,也强调问题定义、机制解释、局限分析与复现细节,以满足毕业论文对研究深度与工作态度的要求。

\subsection{文献脉络与研究定位}
为保证后续方法设计与实验结论具备可比性与可追溯性,本文在文献层面采用“经典基础 + 近年进展 + 前沿融合”的三层梳理框架。经典基础涵盖协同过滤、序列建模与排序学习等工作\cite{koren2009matrix,rendle2009bpr,rendle2010factorizing,grbovic2015commerce,hidasi2016gru4rec};近年进展重点关注Next POI任务中的时空序列与图学习路线\cite{cheng2013where,kong2018hst,sun2020where,luo2021stan,CFPRec};前沿融合聚焦LLM驱动推荐与结构先验注入路线\cite{li2023instructrec,hou2024recgpt,bao2023chatrec,lin2024llmrank,sun2024promptrec}。

面向Next POI这一具体问题,本文重点参考三类证据。第一类是“纯时空/图模型”证据,包括空间门控、超图、对比学习与扩散协同等方向\cite{zhao2019stgn,yang2019lbsn2vec,yang2020flashback,zhou2020sgl,DCPR}。第二类是“语义增强”证据,包括语义城市建模与跨模态地理表示学习\cite{Pasquale2024City,M3PA,CLIP,wav2vec,liu2024mobilityfm}。第三类是“LLM推荐框架”证据,包括检索增强、协同对齐与推理链增强方法\cite{ReLLa,SeCor,GraphGPT,qu2024onerecthink,zhou2023unicorn},以及近两年面向Next POI的最新模型进展\cite{zhuang2024tau,feng2025lrsa,wu2024mrpllm,wu2024mas4poi,li2025rallmpoi}。上述证据共同支持本文采用“结构建模 + 语义增强 + 协同训练”的总体定位。

\section{主要贡献与论文结构安排}
\subsection{主要贡献}
本文主要贡献如下:
\begin{enumerate}
    \item 提出面向Next POI任务的大小模型协同框架,统一时空结构建模与语义推理过程;
    \item 设计时间增强序列动态图与双向转移建模机制,提升小模型对复杂出行规律的表达能力;
    \item 设计地理坐标注入模块(GCIM)与POI对齐模块(PAM),缓解LLM空间幻觉并增强跨场景泛化;
    \item 构建围绕研究问题的实验评估流程,从总体性能、模块贡献、协同有效性与效率开销四个维度验证方法有效性。
\end{enumerate}

\subsection{论文结构安排}
全文共五章,组织如下:
\begin{enumerate}
    \item 第1章为绪论,介绍研究背景、核心挑战、技术路线与主要贡献;
    \item 第2章为相关研究与问题分析,给出统一任务定义与符号约定,综述相关方法并凝练研究空白;
    \item 第3章为方法章,详细阐述小模型分支、大模型分支及协同训练机制;
    \item 第4章为实验章,按研究问题组织实验设置、结果分析与机制诊断;
    \item 第5章为结论与展望,总结全文并讨论后续研究方向。
\end{enumerate}

\section{本章小结}
本章从应用背景与学术问题两个层面阐明了Next POI研究的必要性,并围绕任务特点提炼了时间异质性、空间连续性、转移先验注入和协同优化四类关键挑战。针对上述问题,本文给出了“结构先验建模 + 语义推理增强 + 协同对齐训练”的总体思路,明确了主要贡献与后续章节分工。下一章将进一步通过文献综述与问题分析,建立本文方法设计的理论与经验依据。
