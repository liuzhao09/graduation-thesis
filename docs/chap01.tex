\chapter{绪论}
\label{chap:intro}

\section{本章引言}
本章作为全文的总体导引,主要回答“为什么做、做什么、如何做、做出了什么”四个核心问题。首先从推荐系统向线下时空决策延展的现实背景出发,说明Next POI任务的研究价值;随后给出本文关注的问题边界与核心挑战,明确现有方法在时间异质性、空间一致性和转移先验注入方面的不足;在此基础上概述本文的大小模型协同技术路线、主要创新点与章节组织关系,为后续相关研究、方法细节和实验验证建立统一叙事主线。

\section{研究背景}
在数字经济快速发展的背景下,推荐系统已成为互联网平台的核心基础能力,并持续重塑用户的信息获取与消费决策方式。在电商场景中,阿里巴巴、拼多多、京东、亚马逊等平台通过个性化推荐连接“人-货-场”,显著提升了商品发现效率与交易转化;在内容分发场景中,抖音、小红书、快手等平台依托推荐机制完成兴趣匹配,深刻影响用户的注意力分配与内容消费习惯。可以看到,推荐系统已从单一功能模块演进为平台竞争力的关键基础设施。

随着推荐范式由“线上内容匹配”逐步扩展到“线下服务决策支持”,位置感知推荐的重要性持续上升。以美团、大众点评、滴滴等本地生活与出行平台为例,系统不仅需要回答“用户喜欢什么”,还需要在具体时空约束下回答“用户此刻去哪里、下一步可能前往何处”。在这一过程中,地理位置已由辅助特征转化为核心变量:其既影响候选集合的可达边界,也直接决定推荐结果的时效性与可执行性。因此,下一兴趣点推荐(Next Point-of-Interest Recommendation, Next POI)成为连接推荐系统、地理信息建模与城市计算的重要研究问题。

与传统项目推荐相比,Next POI任务同时受地理可达性、时间节律、活动语义与行为路径连续性的共同约束。随着智能手机、移动互联网和位置服务平台的普及,用户在日常生活中持续产生大规模时空行为数据;以签到轨迹为代表的记录不仅包含“去过哪里”,还隐含“何时去、从哪里来、下一步去哪”的动态规律。这为该任务提供了坚实的数据基础,也使其在学术研究与工程应用层面都具有突出价值。

\section{研究意义}
本文工作的意义主要体现在理论与应用两个层面。
\begin{enumerate}
    \item \textbf{理论意义}:Next POI任务位于“时空序列建模 + 图结构学习 + 语言语义推理”的交叉区域。围绕该任务构建统一框架,有助于回答一个更普遍的问题:在复杂真实场景中,结构归纳偏置与大模型语义能力应如何协同,而不是相互替代。
    \item \textbf{方法意义}:现有方法往往在单一维度做到较优,例如序列方法在局部时序上较强、图方法在高阶关系上较强、LLM在语义泛化上较强。本文尝试给出一条工程可落地的融合路径,使“结构建模能力”与“语义推理能力”在同一训练框架内形成互补。
    \item \textbf{应用意义}:在本地生活、城市出行和文旅推荐等场景中,系统需要在有限时空约束下提供可执行推荐。若预测结果与真实地理规律不一致,即使语义上“看起来合理”,也难以形成有效服务。本文强调的空间一致性与转移可解释性,直接关系到系统的可用性与用户信任。
\end{enumerate}

\section{研究问题分解}
为避免“问题过大、方法过散”,本文将总体目标分解为五个可验证子问题,并在后文通过RQ体系进行闭环验证:
\begin{enumerate}
    \item \textbf{P1:时段异质性建模}。同一POI在不同时间段具有不同的转移规律,如何以低额外成本显式建模该异质性;
    \item \textbf{P2:双向转移表达}。下一跳预测不仅取决于“当前点指向谁”,也取决于“目标点通常从哪里来”,如何同时表达转出与转入偏好;
    \item \textbf{P3:地理连续性注入}。LLM语义空间不天然满足地理邻近关系,如何将经纬度与空间层级先验注入语义编码流程;
    \item \textbf{P4:结构先验跨空间对齐}。图模型学得的POI关系如何映射到LLM语义空间,并在推理阶段持续发挥作用;
    \item \textbf{P5:协同训练可部署}。在效果提升之外,如何保证训练成本、推理时延与参数规模仍处于可接受范围。
\end{enumerate}
通过上述分解,本文将“是否有效”与“为什么有效”区分处理,从而避免仅凭主结果表格给出结论。

\section{问题陈述}
本文关注的核心问题是:给定用户历史签到轨迹及其时间、空间和语义上下文,预测用户下一时刻最可能访问的POI,并输出Top-$K$候选列表。该问题兼具序列预测与结构推断属性,要求模型同时具备局部时空建模能力与跨场景泛化能力。

本文不在绪论中展开完整符号体系与公式定义,统一的任务定义与符号约定放在第\ref{chap:related}章给出,作为全文的单一依据。

\section{核心挑战}
尽管现有研究已在时空推荐方向取得进展,面向真实出行场景的Next POI建模仍面临以下挑战:
\begin{enumerate}
    \item \textbf{时间异质性挑战}:同一POI在不同时间段的转移模式差异显著,统一转移机制容易造成建模偏差;
    \item \textbf{空间连续性挑战}:经纬度与语义表示空间之间缺乏天然同构,纯文本建模容易产生地理不一致预测;
    \item \textbf{转移先验注入挑战}:仅依赖文本上下文时,模型难以充分利用POI图中的高阶转移关系;
    \item \textbf{协同优化挑战}:小模型结构先验与大模型语义能力如何在统一框架内稳定协同,仍缺乏系统方案。
\end{enumerate}

\begin{figure}[htbp]
    \centering
    \includegraphics[width=0.92\textwidth]{fig1_GCIM_c1.pdf}
    \caption{空间连续性挑战示意:原始坐标文本难以稳定保持地理邻近关系,导致语义空间与地理空间错位。}
    \label{fig:c1-gcim-motivation}
\end{figure}

如图\ref{fig:c1-gcim-motivation}所示,直接将坐标作为文本输入时,模型难以在语义空间中保持真实地理邻近结构。该现象说明Next POI任务需要专门的地理编码机制,而非仅依赖通用文本建模。基于这一挑战,本文在第\ref{chap:method}章设计GCIM模块以增强空间一致性。

\begin{figure}[htbp]
    \centering
    \includegraphics[width=0.92\textwidth]{fig2_PAM_c2.pdf}
    \caption{转移先验注入挑战示意:仅依赖token语义难以恢复POI图中的高阶迁移关系。}
    \label{fig:c1-pam-motivation}
\end{figure}

如图\ref{fig:c1-pam-motivation}所示,若缺少结构化对齐路径,模型对“未显式出现但高转移概率”的候选点识别能力不足。该现象说明POI关系先验需要显式注入语义空间,而非完全依赖隐式学习。基于这一挑战,本文在第\ref{chap:method}章设计PAM模块以提升转移感知能力。

\section{研究思路与技术路线}
针对上述挑战,本文采用“大小模型协同学习”的总体路线:
\begin{enumerate}
    \item 在小模型侧构建时间增强序列动态图与双向转移机制,学习稳定的时空结构先验;
    \item 在大模型侧设计地理坐标注入模块与POI对齐模块,增强LLM的空间一致性与转移感知能力;
    \item 通过嵌入对齐与两阶段训练,将结构知识与语义推理能力融合到统一框架中。
\end{enumerate}

该路线的目标不是简单叠加模型,而是在可部署约束下实现“结构归纳偏置 + 语义泛化能力”的互补增益。

\section{研究内容与工作安排}
围绕上述路线,本文研究内容可进一步细化为四项工作:
\begin{enumerate}
    \item \textbf{任务分析与基线整理}:系统梳理Next POI领域中的序列、图与LLM三类方法,统一问题定义、指标解释与实验比较边界;
    \item \textbf{小模型机制设计}:围绕时间异质性与迁移方向性,设计TSDG与双向转移建模,并通过动态图权重增强复杂场景区分能力;
    \item \textbf{大模型增强设计}:围绕地理连续性与转移先验缺失,设计GCIM与PAM模块,将结构先验以可训练方式注入LLM;
    \item \textbf{实验验证与诊断分析}:基于RQ组织实验证据,覆盖主结果、消融、误差类型、效率开销与部署讨论,形成“设置-结果-解释”闭环。
\end{enumerate}
从论文完成度角度,本文不仅给出模型结果,也强调问题定义、机制解释、局限分析与复现细节,以满足毕业论文对研究深度与工作态度的要求。

\section{主要贡献}
本文主要贡献如下:
\begin{enumerate}
    \item 提出面向Next POI任务的大小模型协同框架,统一时空结构建模与语义推理过程;
    \item 设计时间增强序列动态图与双向转移建模机制,提升小模型对复杂出行规律的表达能力;
    \item 设计地理坐标注入模块(GCIM)与POI对齐模块(PAM),缓解LLM空间幻觉并增强跨场景泛化;
    \item 构建围绕研究问题的实验评估流程,从总体性能、模块贡献、协同有效性与效率开销四个维度验证方法有效性。
\end{enumerate}

\section{论文结构安排}
全文共五章,组织如下:
\begin{enumerate}
    \item 第1章为绪论,介绍研究背景、核心挑战、技术路线与主要贡献;
    \item 第2章为相关研究与问题分析,给出统一任务定义与符号约定,综述相关方法并凝练研究空白;
    \item 第3章为方法章,详细阐述小模型分支、大模型分支及协同训练机制;
    \item 第4章为实验章,按研究问题组织实验设置、结果分析与机制诊断;
    \item 第5章为结论与展望,总结全文并讨论后续研究方向。
\end{enumerate}

\section{本章小结}
本章从应用背景与学术问题两个层面阐明了Next POI研究的必要性,并围绕任务特点提炼了时间异质性、空间连续性、转移先验注入和协同优化四类关键挑战。针对上述问题,本文给出了“结构先验建模 + 语义推理增强 + 协同对齐训练”的总体思路,明确了主要贡献与后续章节分工。下一章将进一步通过文献综述与问题分析,建立本文方法设计的理论与经验依据。
