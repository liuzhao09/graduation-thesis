\chapter{绪论}
\label{chap:intro}

\section{研究背景}
在数字经济快速发展的背景下,推荐系统已成为互联网平台的核心基础能力,并持续重塑用户的信息获取与消费决策方式。在电商场景中,阿里巴巴、拼多多、京东、亚马逊等平台通过个性化推荐连接“人-货-场”,显著提升了商品发现效率与交易转化;在内容分发场景中,抖音、小红书、快手等平台依托推荐机制完成兴趣匹配,深刻影响用户的注意力分配与内容消费习惯。可以看到,推荐系统已从单一功能模块演进为平台竞争力的关键基础设施。

随着推荐范式由“线上内容匹配”逐步扩展到“线下服务决策支持”,位置感知推荐的重要性持续上升。以美团、大众点评、滴滴等本地生活与出行平台为例,系统不仅需要回答“用户喜欢什么”,还需要在具体时空约束下回答“用户此刻去哪里、下一步可能前往何处”。在这一过程中,地理位置已由辅助特征转化为核心变量:其既影响候选集合的可达边界,也直接决定推荐结果的时效性与可执行性。因此,下一兴趣点推荐(Next Point-of-Interest Recommendation, Next POI)成为连接推荐系统、地理信息建模与城市计算的重要研究问题。

与传统项目推荐相比,Next POI任务同时受地理可达性、时间节律、活动语义与行为路径连续性的共同约束。随着智能手机、移动互联网和位置服务平台的普及,用户在日常生活中持续产生大规模时空行为数据;以签到轨迹为代表的记录不仅包含“去过哪里”,还隐含“何时去、从哪里来、下一步去哪”的动态规律。这为该任务提供了坚实的数据基础,也使其在学术研究与工程应用层面都具有突出价值。

\section{问题陈述}
本文关注的核心问题是:给定用户历史签到轨迹及其时间、空间和语义上下文,预测用户下一时刻最可能访问的POI,并输出Top-$K$候选列表。该问题兼具序列预测与结构推断属性,要求模型同时具备局部时空建模能力与跨场景泛化能力。

本文不在绪论中展开完整符号体系与公式定义,统一的任务定义与符号约定放在第\ref{chap:related}章给出,作为全文的单一依据。

\section{核心挑战}
尽管现有研究已在时空推荐方向取得进展,面向真实出行场景的Next POI建模仍面临以下挑战:
\begin{enumerate}
    \item \textbf{时间异质性挑战}:同一POI在不同时间段的转移模式差异显著,统一转移机制容易造成建模偏差;
    \item \textbf{空间连续性挑战}:经纬度与语义表示空间之间缺乏天然同构,纯文本建模容易产生地理不一致预测;
    \item \textbf{转移先验注入挑战}:仅依赖文本上下文时,模型难以充分利用POI图中的高阶转移关系;
    \item \textbf{协同优化挑战}:小模型结构先验与大模型语义能力如何在统一框架内稳定协同,仍缺乏系统方案。
\end{enumerate}

\section{研究思路与技术路线}
针对上述挑战,本文采用“大小模型协同学习”的总体路线:
\begin{enumerate}
    \item 在小模型侧构建时间增强序列动态图与双向转移机制,学习稳定的时空结构先验;
    \item 在大模型侧设计地理坐标注入模块与POI对齐模块,增强LLM的空间一致性与转移感知能力;
    \item 通过嵌入对齐与两阶段训练,将结构知识与语义推理能力融合到统一框架中。
\end{enumerate}

该路线的目标不是简单叠加模型,而是在可部署约束下实现“结构归纳偏置 + 语义泛化能力”的互补增益。

\section{主要贡献}
本文主要贡献如下:
\begin{enumerate}
    \item 提出面向Next POI任务的大小模型协同框架,统一时空结构建模与语义推理过程;
    \item 设计时间增强序列动态图与双向转移建模机制,提升小模型对复杂出行规律的表达能力;
    \item 设计地理坐标注入模块(GCIM)与POI对齐模块(PAM),缓解LLM空间幻觉并增强跨场景泛化;
    \item 构建围绕研究问题的实验评估流程,从总体性能、模块贡献、协同有效性与效率开销四个维度验证方法有效性。
\end{enumerate}

\section{论文结构安排}
全文共五章,组织如下:
\begin{enumerate}
    \item 第1章为绪论,介绍研究背景、核心挑战、技术路线与主要贡献;
    \item 第2章为相关研究与问题分析,给出统一任务定义与符号约定,综述相关方法并凝练研究空白;
    \item 第3章为方法章,详细阐述小模型分支、大模型分支及协同训练机制;
    \item 第4章为实验章,按研究问题组织实验设置、结果分析与机制诊断;
    \item 第5章为结论与展望,总结全文并讨论后续研究方向。
\end{enumerate}
