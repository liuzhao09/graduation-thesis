\chapter{绪论}
\label{chap:intro}

\section{研究背景}
随着智能手机、移动互联网和位置服务平台的普及,用户在日常生活中不断产生大规模时空行为数据。以签到记录为代表的轨迹数据不仅包含“去过哪里”,还隐含“何时去、从哪里来、下一步去哪”的动态规律。这类数据为个性化推荐、出行规划、城市计算和商业选址提供了重要支撑。下一兴趣点推荐(Next Point-of-Interest Recommendation, Next POI)正是在这一背景下形成的关键任务,其目标是根据用户历史轨迹预测下一时刻最可能访问的POI。

从应用层面看,Next POI推荐直接服务于智慧出行、文旅推荐、生活服务平台和城市治理等场景。对于用户而言,高质量推荐可以缩短决策时间并提升体验;对于平台而言,可提升转化效率与用户粘性;对于城市管理者而言,可为交通组织、商圈活力评估和公共资源配置提供数据支持。因此,研究高精度、强鲁棒、可泛化的Next POI推荐方法具有显著理论意义与工程价值。

\section{问题定义与符号约定}
设用户集合为 $\mathcal{U}=\{u_1,u_2,\ldots,u_M\}$,POI集合为 $\mathcal{L}=\{\ell_1,\ell_2,\ldots,\ell_N\}$。用户 $u$ 的第 $i$ 条签到记为
\begin{equation}
x_i=(u,\ell_i,t_i,g_i,c_i),
\label{eq:c1-checkin}
\end{equation}
对应式(\ref{eq:c1-checkin}),其中 $\ell_i$ 为POI标识,$t_i$ 为时间戳,$g_i=(lat_i,lon_i)$ 为地理坐标,$c_i$ 为类别语义。用户历史轨迹表示为
\begin{equation}
\mathcal{T}_u=\{x_1,x_2,\ldots,x_n\},\quad t_1<t_2<\cdots<t_n.
\label{eq:c1-traj}
\end{equation}
式(\ref{eq:c1-traj})给出时间有序的行为序列定义。

Next POI推荐任务可表示为学习映射函数
\begin{equation}
f:\mathcal{T}_u\mapsto \hat{\ell}_{n+1},\quad \hat{\ell}_{n+1}\in\mathcal{L},
\label{eq:c1-task}
\end{equation}
即在式(\ref{eq:c1-task})框架下,使真实下一POI $\ell_{n+1}$ 在候选列表中排名尽可能靠前。若输出Top-$K$列表,则记为 $\hat{\mathbf{y}}_u=[\hat{\ell}^{(1)},\hat{\ell}^{(2)},\ldots,\hat{\ell}^{(K)}]$。

与传统静态推荐不同,Next POI任务需同时建模:
\begin{enumerate}
    \item 时间依赖:工作日/周末、早晚高峰等时段模式;
    \item 空间依赖:邻近可达性、区域功能分布与路径连续性;
    \item 语义依赖:POI类别、场景语义与用户个体偏好;
    \item 结构依赖:跨用户共享的POI转移图与高阶关系。
\end{enumerate}

\section{推荐系统与序列建模基础}
为便于后文展开,本节简要给出本文涉及的核心建模基础。

\subsection{矩阵分解与隐反馈学习基础}
在推荐系统中,常以用户-项目交互矩阵 $\mathbf{R}\in\mathbb{R}^{M\times N}$ 建模偏好。矩阵分解方法将其近似为
\begin{equation}
\mathbf{R}\approx \mathbf{P}\mathbf{Q}^{\top},
\label{eq:c1-mf}
\end{equation}
式(\ref{eq:c1-mf})中,$\mathbf{P}\in\mathbb{R}^{M\times d}$ 为用户潜向量,$\mathbf{Q}\in\mathbb{R}^{N\times d}$ 为POI潜向量。对应目标函数可写为
\begin{equation}
\min_{\mathbf{P},\mathbf{Q}}\sum_{(u,\ell)\in\Omega}(r_{u\ell}-\mathbf{p}_u^{\top}\mathbf{q}_{\ell})^2+\lambda(\|\mathbf{P}\|_F^2+\|\mathbf{Q}\|_F^2).
\label{eq:c1-mf-obj}
\end{equation}
式(\ref{eq:c1-mf-obj})刻画了重构误差与参数正则化的平衡。

在隐反馈场景中,常采用BPR排序损失:
\begin{equation}
\mathcal{L}_{\text{BPR}}=-\sum_{(u,i,j)}\log\sigma(\hat{y}_{u,i}-\hat{y}_{u,j})+\lambda\|\Theta\|_2^2,
\label{eq:c1-bpr}
\end{equation}
见式(\ref{eq:c1-bpr}),其中 $i$ 为正样本,$j$ 为负样本。FPMC~\cite{FPMC} 与 PRME~\cite{PRME} 均可视为这一路线在时序推荐中的代表。

\subsection{时序与注意力建模基础}
基于RNN的方法通过递归更新隐状态:
\begin{equation}
\mathbf{h}_t=\phi(\mathbf{W}_x\mathbf{x}_t+\mathbf{W}_h\mathbf{h}_{t-1}+\mathbf{b}),
\label{eq:c1-rnn}
\end{equation}
其形式见式(\ref{eq:c1-rnn}),并以 $\mathbf{h}_t$ 预测下一项。LSTM~\cite{LSTM} 通过门控机制缓解长依赖梯度问题,ST-RNN~\cite{ST-RNN}、HST-LSTM~\cite{HST-LSTM} 在此基础上引入时空上下文。

Transformer注意力机制将序列表示为
\begin{equation}
\text{Attention}(\mathbf{Q},\mathbf{K},\mathbf{V})=\text{Softmax}\left(\frac{\mathbf{Q}\mathbf{K}^{\top}}{\sqrt{d}}\right)\mathbf{V},
\label{eq:c1-attn}
\end{equation}
如式(\ref{eq:c1-attn})所示,用于捕获长程依赖。GETNext~\cite{GETNext} 将轨迹流图与Transformer结合,显著提升了复杂迁移场景下的预测能力。

\subsection{图神经网络基础}
对于图 $\mathcal{G}=(\mathcal{V},\mathcal{E})$,GNN典型消息传递过程为
\begin{equation}
\mathbf{h}_v^{(l+1)}=\psi\Big(\mathbf{h}_v^{(l)},\,\textstyle\bigoplus_{u\in\mathcal{N}(v)}\phi(\mathbf{h}_u^{(l)},\mathbf{h}_v^{(l)},\mathbf{e}_{uv})\Big),
\label{eq:c1-gnn}
\end{equation}
即式(\ref{eq:c1-gnn}),其中 $\bigoplus$ 表示聚合算子。基于归一化邻接矩阵的GCN形式可写为
\begin{equation}
\mathbf{H}^{(l+1)}=\sigma\left(\tilde{\mathbf{D}}^{-1/2}\tilde{\mathbf{A}}\tilde{\mathbf{D}}^{-1/2}\mathbf{H}^{(l)}\mathbf{W}^{(l)}\right).
\label{eq:c1-gcn}
\end{equation}
见式(\ref{eq:c1-gcn})。

Next POI场景中,图结构可来自POI转移、用户-POI交互或超图关系。STHGCN~\cite{STHGCN}、GraphFlashback~\cite{GraphFlashback} 和 SNPM~\cite{SNPM} 证明了图结构对稀疏场景和跨用户迁移的显著价值。

\section{大语言模型引入与核心挑战}
LLM在语义理解和上下文推理方面表现突出,推动了推荐系统从“结构拟合”向“语义推理+结构约束”转变。LLM4POI~\cite{LLM4POI} 率先将Next POI任务转化为提示学习与生成问题,验证了大模型在冷启动与语义泛化方面的潜力。GraphGPT~\cite{GraphGPT}、LLaRA~\cite{LLaRA}、E4SRec~\cite{E4SRec}、CoLLM~\cite{CoLLM}、LLMRec~\cite{LLMRec} 等研究进一步说明,大模型可有效吸收图结构与协同信号。

然而,LLM直接用于Next POI仍面临两类关键难题:

\textbf{挑战C1:空间感知能力不足。}
高精度经纬度在LLM分词后常形成长token序列,既增加计算成本,又破坏地理邻近性表达。由于语义空间与地理空间缺乏天然同构,模型可能出现“语义相近但地理很远”的误判,甚至生成不合理POI。相关城市基础模型研究也指出,地理连续空间建模需要专门编码机制~\cite{Pasquale2024City,Jonathan2023Gpt}。

\textbf{挑战C2:POI转移先验注入不足。}
纯文本提示对“轨迹中的图结构转移规律”刻画有限。当目标POI未显式出现在历史输入时,模型容易回退到高频历史点,难以恢复真实转移关系。GeoSAN~\cite{GeoSAN}、ROTAN~\cite{ROTAN}、MTNet~\cite{MTNet} 和 CFPRec~\cite{CFPRec} 表明,显式引入转移结构对提升预测稳定性至关重要。

\section{研究思路与技术路线}
为应对上述问题,本文采用“大小模型协同学习”路线:
\begin{enumerate}
    \item 在小模型侧构建时间增强序列动态图与双向转移建模,学习高质量时空表示;
    \item 在大模型侧构建地理坐标注入模块与POI对齐模块,将空间连续性与转移先验注入LLM;
    \item 通过两阶段训练与参数高效微调,实现结构知识与语义知识协同优化。
\end{enumerate}

该路线并非简单模型拼接,而是“结构归纳偏置”与“语义生成能力”的互补融合:小模型提供稳定时空先验,大模型提供语义泛化与上下文推理能力,共同提升准确率、鲁棒性与可迁移性。

\section{主要研究内容与创新点}
围绕上述技术路线,本文工作可归纳为以下三点:
\begin{enumerate}
    \item 提出面向Next POI任务的大小模型协同框架,统一时空结构建模与语义推理;
    \item 提出时间槽偏好与双向转移建模机制,提升小模型对复杂出行规律的表达能力;
    \item 设计地理坐标注入与POI对齐机制,缓解LLM空间幻觉并增强跨城冷启动泛化。
\end{enumerate}

\section{论文结构安排}
全文共五章,具体安排如下:
\begin{enumerate}
    \item 第1章为绪论,介绍研究背景、问题定义、核心挑战与研究路线;
    \item 第2章为相关研究与问题分析,系统梳理小模型与大模型方法并明确研究空白;
    \item 第3章为方法章,详细介绍小模型分支、大模型分支及协同训练机制;
    \item 第4章为实验章,从总体性能、消融、鲁棒性与效率角度验证方法有效性;
    \item 第5章为结论与展望,总结工作贡献并讨论未来研究方向。
\end{enumerate}
