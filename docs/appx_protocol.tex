\chapter{实验执行日志与复现流程补充}
\label{appx:protocol}

本附录给出实验执行过程中的流程化记录模板与复现实践建议,覆盖数据、训练、评估、诊断、报告五个环节。该部分用于增强论文工作的透明度与可核查性。

\section{实验生命周期管理}
\subsection{阶段划分}
建议将实验分为五个阶段:
\begin{enumerate}
    \item 数据准备阶段;
    \item 基线复现阶段;
    \item 模块开发与单测阶段;
    \item 联合训练与调参阶段;
    \item 结果固化与报告阶段。
\end{enumerate}
各阶段应设里程碑与退出条件,避免“边写代码边改评估”导致流程混乱。

\subsection{版本管理建议}
每次实验必须记录代码版本、配置版本、数据版本。建议采用统一命名规范:
\begin{equation}
\texttt{exp\_\{date\}\_\{dataset\}\_\{model\}\_\{seed\}}.
\end{equation}
该规范便于追溯同一结果的生成路径。

\section{数据层复现清单}
\subsection{原始数据记录}
记录数据来源链接、下载时间、文件校验值、原始字段说明。若数据有更新,应保留历史版本以支持可重复实验。

\subsection{预处理日志}
预处理日志至少包含:过滤规则、阈值、异常样本数量、最终样本统计。建议输出结构化日志文件,避免仅通过终端打印保存。

\subsection{切分与采样记录}
记录训练/验证/测试切分索引、负采样策略与随机种子。对关键实验,建议将样本索引固化到文件中。

\section{训练层复现清单}
\subsection{配置固化}
每个实验配置应至少包含:模型超参数、优化器参数、训练轮数、早停条件、损失权重。建议以YAML或JSON保存,并在结果目录中备份。

\subsection{训练日志字段}
建议统一记录以下字段:
\begin{enumerate}
    \item 当前epoch与step;
    \item 训练损失与验证损失;
    \item 主指标(Acc@1、MRR等);
    \item 学习率与梯度范数;
    \item 显存占用与单步耗时。
\end{enumerate}
这些字段可用于诊断收敛异常和资源瓶颈。

\subsection{异常处理流程}
当出现损失爆炸、梯度异常或训练中断时,应保留异常前后日志、检查点与配置快照,并记录处理动作(如降低学习率、恢复检查点、调整损失权重)。

\section{评估层复现清单}
\subsection{指标实现一致性}
所有模型应共用同一评估脚本与同一候选集构造规则。若不同模型需特殊处理,应在脚本中显式分支并记录原因。

\subsection{多次重复策略}
关键结论建议至少在多个随机种子下重复,以观察稳定趋势。报告应优先展示平均表现与相对趋势,避免只展示单次最优值。

\subsection{统计检验建议}
对核心对比可采用配对统计检验判断显著性。应报告检验对象、样本量与显著性阈值设置,提升结论可信度。

\section{诊断层复现清单}
\subsection{消融实验组织}
建议采用“单变量变化”原则:每次仅移除或替换一个模块,其他设置保持一致。该策略可避免多因素耦合造成结论歧义。

\subsection{错误分析模板}
建议将错误按类型归档:远跳错误、语义误判、历史重复、跨区漏判。每类错误保留样本示例,记录模型输出与真实标签,便于后续定位问题。

\subsection{可视化记录}
图表应保存原始绘图数据与脚本,避免“图有结论但无法复现”。同时建议保存高分辨率版本用于论文排版。

\section{报告层规范建议}
\subsection{表述规范}
论文正文建议采用“现象-解释-回扣设计”的三步写法:先描述观察,再解释机制,最后关联方法模块与RQ。这可显著提升论证完整性。

\subsection{结果边界说明}
对于未覆盖场景或表现不稳定情况,应主动说明边界与可能原因。这种“正反结果并陈”的写法更符合学术规范,也更能体现研究态度。

\subsection{附录材料管理}
将配置、日志、补充图表、案例分析集中放入附录,可在不打断主线叙事的前提下完整呈现研究投入与证据链。

\section{示例化实验日志模板}
以下给出简化日志模板:
\begin{enumerate}
    \item 实验编号:
    \item 代码版本:
    \item 数据版本:
    \item 训练配置摘要:
    \item 最佳检查点路径:
    \item 主指标结果:
    \item 消融对比结果:
    \item 异常记录与处理:
    \item 复现实验结论:
\end{enumerate}
该模板可作为课题组长期实验管理文档的基础。

\section{经验总结}
通过完整执行上述流程,可以显著降低实验不可复现、结果难追溯、结论难解释等常见问题。对于毕业论文而言,这不仅提升内容体量,更重要的是体现研究过程的规范性、透明性与可验证性。
