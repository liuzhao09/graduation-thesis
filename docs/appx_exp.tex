\chapter{实验补充分析与案例讨论}
\label{appx:exp}

本附录补充第\ref{chap:exp}章未展开的实验分析内容,包含分群评估、误差归因、参数敏感性与案例讨论,用于增强实证结论的完整性。

\section{分群评估补充}
\subsection{按轨迹长度分群}
将测试样本按历史长度划分为短轨迹、中轨迹与长轨迹三组。观察表明,协同模型在短轨迹场景下相对增益更明显,说明语义分支在信息不足时提供了有效补偿;在长轨迹场景下,小模型结构分支可稳定利用历史规律,维持较高上限性能。

\subsection{按地理跨度分群}
按真实下一跳距离将样本划分为近距离、中距离、远距离三组。结果趋势显示,GCIM引入后中远距离样本的远跳误判下降,说明地理连续性建模对跨区域迁移预测具有直接贡献。

\subsection{按POI流行度分群}
将目标POI按访问频次划分为头部、腰部、长尾。协同框架在长尾组保持更稳定表现,反映PAM对结构先验的注入可缓解纯语义模型在长尾区域的召回不足问题。

\section{参数敏感性补充}
\subsection{融合系数 $\eta$}
当 $\eta$ 偏大时,模型更依赖LLM语义分支,通常有利于冷启动样本;当 $\eta$ 偏小时,模型更依赖结构分支,通常有利于高频规律场景。中间区间通常能在精度与稳定性上取得平衡。

\subsection{时间槽数 $Z$}
时间槽过少会使行为节律被过度平均,过多则产生稀疏与噪声。实验趋势验证了第\ref{chap:method}章关于“中等粒度优先”的设计判断。

\subsection{LoRA秩}
LoRA秩过低会限制表达能力,过高会增加训练成本并带来过拟合风险。参数高效微调的最佳点通常需要在验证集上按数据规模与任务复杂度共同选择。

\section{误差归因补充}
\subsection{语义相关但地理不合理}
该类错误常见于纯文本提示场景:模型能识别活动语义,却忽略地理可达性。GCIM可降低此类错误比例,说明地理编码对语义空间具有校正作用。

\subsection{地理邻近但转移概率低}
该类错误反映模型只关注空间邻近而忽略行为路径规律。PAM引入后,该类错误有所下降,说明结构迁移先验有助于纠正“近邻即正确”的简化假设。

\subsection{历史重复偏置}
模型可能过度偏好近期高频点而忽略用户阶段性意图变化。双向转移与序列聚合改进后,该类错误下降,说明结构分支对路径方向信息的建模具有实效。

\section{案例讨论}
\subsection{典型通勤样本}
在“居住区$\rightarrow$办公区$\rightarrow$餐饮区”样本中,协同模型能够在午间时段优先推荐地理可达且语义匹配的餐饮POI,而非简单重复办公区附近高频点,体现了时段感知与语义理解的协同作用。

\subsection{周末休闲样本}
在“商圈漫游”样本中,纯结构模型倾向保守推荐已访问点,纯语义模型可能推荐语义相关但距离较远点;协同模型通常能给出“类别匹配 + 空间可达”候选,排序更稳定。

\subsection{跨区域迁移样本}
当用户由工作区转向新区活动时,若缺少地理与结构双重约束,模型容易滞留在旧区域。协同模型在此类样本中表现更好,说明其具备一定跨区域迁移识别能力。

\section{实验公平性补充说明}
为保证比较公平,本文遵循以下原则:
\begin{enumerate}
    \item 所有方法使用统一数据切分与统一评估脚本;
    \item 指标计算方式一致,候选集合构造规则一致;
    \item 对比方法尽量采用公开实现与推荐配置;
    \item 对关键结果进行重复实验,报告趋势而非单点最优。
\end{enumerate}

\section{负面结果与经验总结}
研究过程中也观察到若干负面现象:
\begin{enumerate}
    \item 过深的对齐层并不总是带来收益,反而可能引入优化不稳定;
    \item 在样本极稀疏场景中,复杂模块叠加可能过拟合,需要更强正则;
    \item 若提示模板频繁变化,会导致训练-推理分布偏移并降低稳定性。
\end{enumerate}
这些负面结果提示:模型改进不应仅追求复杂度提升,更应关注目标一致性、训练稳定性与部署可控性。

\section{小结}
本附录从分群、敏感性、误差归因与案例四个维度补充了主实验结论,进一步说明本文方法在不同样本类型与不同部署约束下的行为特征。相关分析与第\ref{chap:exp}章主结论一致,即协同框架的收益来源于结构与语义两类能力的互补,而非单一路线的偶然优化。
