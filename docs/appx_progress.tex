\chapter{研究过程记录与阶段性成果补充}
\label{appx:progress}

本附录从研究过程视角补充论文工作量与推进路径,重点说明课题从问题定义到模型落地的阶段性任务、关键决策与经验沉淀。该部分可作为论文答辩时对“研究投入与工作态度”的过程性证据。

\section{研究阶段划分}
本文研究过程可概括为四个阶段:
\begin{enumerate}
    \item 问题识别与文献调研阶段;
    \item 方法原型设计与可行性验证阶段;
    \item 协同框架实现与系统实验阶段;
    \item 论文整理与结果固化阶段。
\end{enumerate}
各阶段并非线性孤立,而是多轮迭代推进。

\section{阶段一:问题识别与文献调研}
\subsection{调研目标}
阶段一核心目标是明确研究边界:到底是做“单模型精调”,还是做“结构与语义协同”。通过对序列、图、LLM三类路线的系统梳理,最终确定本文聚焦于协同框架。

\subsection{调研产出}
该阶段主要产出包括:统一任务定义草案、基线方法清单、研究空白列表、初版技术路线图。该产出直接支撑了第\ref{chap:intro}章与第\ref{chap:related}章的写作。

\section{阶段二:方法原型与可行性验证}
\subsection{原型设计思路}
在原型阶段,先分别实现小模型分支与大模型分支,再尝试最小化耦合方式连接两者。早期实验显示,若直接端到端联合训练,易出现优化不稳定与性能波动,因此逐步形成“两阶段训练”方案。

\subsection{关键试错点}
该阶段经历的典型试错包括:
\begin{enumerate}
    \item 时间槽粒度过细导致样本稀疏;
    \item 过深对齐层导致训练震荡;
    \item 纯文本坐标注入导致地理远跳错误偏多。
\end{enumerate}
上述问题推动了TSDG、GCIM、PAM等模块最终形态的形成。

\section{阶段三:系统实验与诊断分析}
\subsection{实验组织方式}
实验阶段采用“主结果先行、机制诊断跟进、效率评估收束”的组织方式。先确认方法是否有效,再解释为何有效,最后评估代价是否可接受。

\subsection{实验治理措施}
为避免结果漂移,本文在实验治理上执行了统一评估脚本、统一数据切分、配置文件固化与多次重复验证。该措施保证了章节间结论的一致性和可追溯性。

\section{阶段四:论文固化与结构优化}
\subsection{写作结构迭代}
论文结构由“模块堆叠”逐步调整为“问题驱动”叙事:绪论明确问题与贡献,相关工作凝练空白,方法章聚焦机制,实验章围绕RQ闭环验证,结论章总结贡献与不足。

\subsection{图文协同优化}
针对关键图表,本文补充了“图意说明-机制解释-实验回扣”三段式阐述,避免“有图无解释”或“解释与图脱节”的常见问题。

\section{阶段性成果清单}
在全过程中形成的代表性成果包括:
\begin{enumerate}
    \item 统一的协同框架定义与模块边界;
    \item 小模型侧时间增强与双向转移机制;
    \item 大模型侧地理注入与POI对齐机制;
    \item 按研究问题组织的实验评估闭环;
    \item 面向复现与答辩的附录化支撑材料。
\end{enumerate}

\section{经验与反思}
\subsection{经验一:先解决可读性,再优化性能}
跨空间协同任务中,表示是否可读比模型是否复杂更重要。先完成对齐再联合优化,通常比一次性端到端更稳定。

\subsection{经验二:指标提升必须配套机制解释}
仅给出主结果表格不足以支撑论文说服力。通过消融、误差分析与案例讨论解释性能来源,是高质量毕业论文的重要组成部分。

\subsection{经验三:工程可部署性应前置考虑}
若方法只在离线指标上提升但部署代价过高,学术与工程价值都会受限。本文在设计初期就引入参数高效微调与可控重排序策略,降低了后期落地阻力。

\section{后续工作计划补充}
围绕本文现有成果,后续可继续推进以下方向:
\begin{enumerate}
    \item 引入在线反馈闭环,研究动态对齐与增量更新机制;
    \item 扩展多模态信息(图文、交通、事件)增强上下文理解;
    \item 推进跨城市迁移评估标准化,提升结果横向可比性;
    \item 构建更贴近业务收益的综合评价指标体系。
\end{enumerate}

\section{附录小结}
本附录从研究过程层面补充了论文工作量与方法演进路径,强调本文工作并非单次实验结果堆叠,而是经过多阶段调研、试错、优化与固化形成的系统研究成果。这一过程性记录可与正文中的方法与实验结论相互印证。
