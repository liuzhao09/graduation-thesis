\chapter{相关研究与问题分析}
\label{chap:related}
\newcommand{\cTwoTblStyle}{\centering\zihao{5}\songti\rmfamily\setlength{\tabcolsep}{4.0pt}\renewcommand{\arraystretch}{1.16}}

\section{本章引言}
本章目标是建立全文统一的问题分析基线,回答“已有研究做到什么程度、仍缺什么、本文据此如何建模”三个问题。具体而言,本章先系统综述序列/图/大模型三条技术路线并提炼能力边界,再在此基础上给出全文统一的任务定义、符号体系与评价口径,最后将文献证据收敛为可执行的研究空白(Gap)和建模原则(DP),为第\ref{chap:method}章的方法设计提供直接依据。

\section{相关工作综述与研究进展}
\subsection{小模型路线综述:序列与图方法}
\subsubsection{序列建模方法}
FPMC~\cite{FPMC}、PRME~\cite{PRME} 等方法以“偏好建模 + 转移建模”为核心;ST-RNN~\cite{ST-RNN}、HST-LSTM~\cite{HST-LSTM}、LSTPM~\cite{LSTPM}、STAN~\cite{STAN} 将时空上下文与注意力机制引入序列编码;CLSPRec~\cite{CLSPRec}、FHCRec~\cite{FHCRec} 进一步通过对比学习提升稀疏场景鲁棒性。

\textbf{局限与启示:}序列模型对局部行为刻画精细,但在时间异质性显式建模和跨用户高阶迁移利用上仍不足,提示本文需引入“时间分段建模 + 结构化转移先验”。

\topichead{进一步讨论}
序列路线的另一个共性问题是对“远距离兴趣跳转”的解释不足。仅凭短期历史时,模型容易将预测压向高频近邻点,导致在跨区域通勤、目的性出行等场景中出现系统性偏差。这一现象提示我们:仅提升序列编码器深度并不能自然解决可达性与结构先验问题,仍需显式引入图结构或地理约束。

\subsubsection{图建模方法}
GETNext~\cite{GETNext}、GraphFlashback~\cite{GraphFlashback}、STHGCN~\cite{STHGCN}、SNPM~\cite{SNPM} 等方法通过POI图或异构图学习高阶关系,缓解稀疏监督问题;ROTAN~\cite{ROTAN}、MTNet~\cite{MTNet} 则强化时间动态建模。

\textbf{局限与启示:}图方法在结构学习上表现突出,但动态图更新成本和异构信息融合复杂度较高,提示本文需在表达能力与可部署性之间做轻量平衡。

\topichead{进一步讨论}
图路线通常依赖全局邻接关系学习高阶迁移,这一优势在数据充足时十分明显,但也带来两个工程问题:其一,动态图重构与邻居采样在大规模场景中开销较高;其二,图表示与语言语义空间缺乏天然对齐,导致与LLM协作时存在“信息可用但难注入”的鸿沟。本文后续PAM模块即针对第二个问题给出对齐路径。

\subsection{大模型路线综述:LLM驱动推荐}
\subsubsection{LLM在推荐中的应用}
LLM推荐研究已从“直接提示预测”演化为多条并行技术路线。第一类是\textbf{指令化与参数高效微调路线},代表工作包括 InstructRec~\cite{li2023instructrec}、RecGPT~\cite{hou2024recgpt}、Chat-REC~\cite{bao2023chatrec} 与 TALLRec~\cite{tallrec2023},核心目标是以较小训练代价完成任务迁移,并增强跨场景泛化。第二类是\textbf{协同语义对齐路线},LLMRec~\cite{LLMRec}、CoLLM~\cite{CoLLM}、CoRAL~\cite{CoRAL}、CLLM4Rec~\cite{cllm4rec2024www}、IDGenRec~\cite{idgenrec2024sigir} 与 A-LLMRec~\cite{allmrec2024kdd} 通过结构信号、协同表示或文本化ID缓解“语言空间-行为空间”不匹配问题。第三类是\textbf{显式推理与检索增强路线},ReLLa~\cite{ReLLa}、LLaRA~\cite{LLaRA}、OneRec~\cite{qu2024onerec}、OneRec-Think~\cite{qu2024onerecthink}、ThinkRec~\cite{wu2024thinkrec}、Agent4Rec~\cite{yao2024agent4rec}、LLMRank~\cite{lin2024llmrank}、MEMO~\cite{fan2024memo} 与 PromptRec~\cite{sun2024promptrec} 强调可追踪中间推断、候选重排稳定性与错误可诊断性。

从建模粒度看,最新工作进一步从“单步生成答案”转向“分阶段推荐流程”。其中,一类方法将\textbf{召回-排序-解释}统一为共享生成接口,以减少模块割裂;另一类方法引入\textbf{外部记忆与检索},在长尾物品和冷启动样本上补充证据链,抑制幻觉与不一致。总体上,前沿演化方向已经由“能不能生成”转向“生成是否可控、可解释、可部署”。

对本研究问题而言,上述进展提供了三点直接启示:其一,仅依赖自然语言提示难以稳定承载结构迁移先验;其二,表示对齐应覆盖“ID语义-行为图结构-上下文意图”三类信息;其三,推理过程需要可回溯,以支持后续误差归因与工程调参。这些认识为后文方法章节的模块化设计提供了相关工作层面的依据。

\subsubsection{LLM在Next POI中的进展}
在Next POI方向,LLM4POI~\cite{LLM4POI} 首先验证了将轨迹预测转化为生成任务的可行性;SeCor~\cite{SeCor} 通过语义-协同表示对齐改进时空序列预测质量;Geo-LLMRec~\cite{wang2024georecllm} 与 POI-LLM基准研究~\cite{zheng2024poillm} 进一步揭示了地理一致性不足、远跳误差偏高与空间幻觉等核心瓶颈。

沿着上述问题,近期工作开始向“地理约束显式化”与“推理过程结构化”推进:如隐私保持的多任务反思机制~\cite{wu2024mrpllm}、多智能体协同推断框架~\cite{wu2024mas4poi}、检索增强地理重排策略~\cite{li2025rallmpoi}。这些方法的共同特点是将“坐标连续性、可达性约束、轨迹方向性”作为显式决策因素,而非完全交由黑盒语义生成隐式学习。

整体上看,LLM驱动的Next POI研究正在经历从“生成可行性验证”到“可控生成与鲁棒部署”的阶段转换。这一趋势也说明:面向真实城市场景,方法评估不能只看命中率,还应同步关注地理误差、跨时段稳定性与推理时延等工程指标。

\subsubsection{标识构造、语义对齐与受约束生成}
近期工作可概括为“标识构造-语义对齐-受约束生成”三阶段。其一,在标识构造层,P5~\cite{p5recsys2022} 将推荐任务统一为语言生成范式,CID~\cite{cid2023} 进一步讨论了不同Item ID编码方式对可学习性的影响。其二,在语义对齐层,CLLM4Rec~\cite{cllm4rec2024www}、A-LLMRec~\cite{allmrec2024kdd} 与 IDGenRec~\cite{idgenrec2024sigir} 通过协同信号对齐缓解“ID空间-语言空间”错位。其三,在生成阶段,TALLRec~\cite{tallrec2023}、GenRec~\cite{genrec2023} 与 Tiger~\cite{tiger2023neurips} 强调受约束或可校验生成,以降低非法ID输出与语义漂移。

对Next POI任务而言,这一主线具有直接启发:POI本质是带地理语义的Item ID,若仅把POI当作普通文本token处理,会放大空间不连续和转移先验缺失问题;因此需要在“ID层可表达、语义层可对齐、解码层可约束”三个环节同时设计,这也是本文GCIM与PAM协同建模的理论依据之一。

\textbf{局限与启示:}现有LLM方案普遍面临坐标语义稀疏、空间连续性不足和转移先验注入弱的问题,说明Next POI场景需要“结构先验 + 语义推理”的协同机制,而非纯文本提示。

\subsubsection{小结性对比}
为更清晰地展示三类方法的能力边界,表\ref{tab:c2-compare}给出总结性比较。
\begin{table}[htbp]
    \cTwoTblStyle
    \caption{序列、图与LLM路线能力边界对比}
    \label{tab:c2-compare}
    \begin{tabular}{p{0.13\textwidth}p{0.25\textwidth}p{0.25\textwidth}p{0.29\textwidth}}
        \toprule
        路线 & 主要优势 & 主要短板 & 对本文设计启示 \\
        \midrule
        序列模型 & 局部时序表达强,参数规模相对可控 & 对高阶迁移与跨场景泛化支持有限 & 需要补充结构化转移先验与时段异质性建模 \\
        图模型 & 高阶关系建模能力强,能缓解稀疏监督 & 动态更新成本高,与语义空间融合困难 & 需要轻量动态图建模与可映射表示空间 \\
        LLM模型 & 语义理解与泛化能力强,适合复杂意图 & 地理连续性弱,纯文本坐标表达不稳定 & 需要专门地理编码与POI先验注入机制 \\
        \bottomrule
    \end{tabular}
\end{table}

\subsection{生成式推荐与显式推理研究进展}
随着生成式推荐的发展,研究重点已从“把推荐写成文本生成”逐步转向“让模型显式地进行中间推理”。这一路线强调:在复杂推荐任务中,仅靠隐式注意力难以稳定地执行约束推断,显式推理链、检索增强与结构对齐是提升稳定性的关键。

\subsubsection{从指令调优到统一生成框架}
近年来,LLM推荐研究在范式上呈现三步演化。第一步是将推荐任务指令化并进行参数高效微调,如 InstructRec~\cite{li2023instructrec}、RecGPT~\cite{hou2024recgpt}、TALLRec~\cite{tallrec2023} 与 Chat-REC~\cite{bao2023chatrec}。第二步是将召回、排序与解释统一到单一生成框架,如 LLMRec~\cite{LLMRec}、CoLLM~\cite{CoLLM}、CLLM4Rec~\cite{cllm4rec2024www} 与 CoRAL~\cite{CoRAL}。第三步是在统一框架中加入外部检索、协同对齐与结构先验以缓解幻觉和不一致,如 ReLLa~\cite{ReLLa}、IDGenRec~\cite{idgenrec2024sigir} 与 LLaRA~\cite{LLaRA}。围绕该路线,也出现了排序增强、记忆增强与提示优化等工作,例如 LLMRank~\cite{lin2024llmrank}、MEMO~\cite{fan2024memo}、PromptRec~\cite{sun2024promptrec} 与 RAG增强推荐\cite{wu2024ragrec},并催生了更系统的综述研究\cite{gao2023llm4rs,lin2024genrecsurvey,zhang2024llmrecsurvey,chen2024rsfm}。

\subsubsection{显式推理范式进展}
与“直接生成答案”不同,显式推理范式强调中间决策轨迹,例如候选过滤、约束检查、理由归纳与自我校验。代表性工作显示,将可追溯推理过程纳入推荐推断有助于提升一致性与可解释性\cite{qu2024onerec,qu2024onerecthink,wu2024thinkrec,yao2024agent4rec,xu2024hybridrec}。这类方法的共同点是:模型不只输出最终推荐,还输出可追溯的中间依据,从而降低“结果可用但难解释”的工程风险。

对Next POI任务而言,显式推理的价值主要在三点:其一,可将“距离可达性、时段约束、历史路径方向”显式纳入决策;其二,可将结构先验作为推理证据而非黑盒隐变量;其三,可在错误分析中定位到底是语义误判还是约束违背。这与本文“GCIM + PAM + 协同重排序”的思路在方法论上是一致的,即把不可控隐式学习转化为可诊断的显式推断流程。

\subsubsection{与本文方法的关系}
本文不直接复制通用 CoT 模板,而是采用“结构对齐 + 轻量协同”的技术路线。原因在于:Next POI场景具有强时空约束与低时延要求,若直接引入冗长推理链,可能带来明显时延与鲁棒性波动。相比之下,本文将推理需求压缩为“可学习对齐模块 + 可控融合打分”,在效率和稳定性上更适合部署场景。

\subsection{中文研究脉络与本土场景启示}
中文学术界在推荐系统、序列推荐和图推荐方面已形成较系统研究脉络\cite{liu2024cn1,zhang2023cn2,wang2023cn3,li2024cn4}。在时空行为建模与位置推荐方面,研究重点逐步转向“地理约束 + 行为节律”的联合建模\cite{zhao2022cn5,he2021cn6,sun2022cn7,yang2023cn8,zheng2024cn14}。在大模型推荐与生成式推荐方面,近年的讨论集中于可解释性、RAG与工程落地\cite{chen2024cn9,gao2024cn10,huang2024cn11,xu2024cn12,qian2024cn15},并与城市时空智能应用形成交叉\cite{liu2023cn13}。

这些中文研究对本文有两点直接启示。第一,真实业务中“可解释 + 可部署”通常与“离线最优”同等重要,方法设计不能只追求指标提升。第二,国内LBS与本地生活场景具备高密度、强约束、强实时特征,要求推荐模型在地理一致性与时延之间做更细粒度权衡。本文采用协同框架而非单一路线,正是基于上述工程现实做出的方法选择。

\subsection{扩展文献归纳}
为完整刻画本文研究语境,本文参考了经典协同过滤与排序学习文献\cite{koren2009matrix,rendle2009bpr,rendle2010factorizing,he2017ncf},以及序列推荐主线工作\cite{hidasi2016gru4rec,kang2018sasrec,sun2019bert4rec,zhang2023sequentialsurvey}。在图推荐与自监督学习方面,重点参考了轻量图卷积、知识图增强与图对比学习工作\cite{he2020lightgcn,wang2019kgat,zhou2020sgl,lin2022simgcl,wu2022gnnrecsurvey}。

在时空推荐与Next POI方向,本文重点覆盖了序列、图与混合路线的代表方法\cite{cheng2013where,he2016inferring,FPMC,PRME,ST-RNN},以及近期性能较强的时空图方法\cite{GETNext,GraphFlashback,SNPM,ROTAN,MTNet}。在LLM与基础模型方向,本文参考了 LLM4POI~\cite{LLM4POI}、SeCor~\cite{SeCor}、POI-LLM基准工作~\cite{zheng2024poillm}、Geo-LLMRec~\cite{wang2024georecllm} 与显式推理增强方法~\cite{qu2024onerecthink}。同时,补充引入 P5~\cite{p5recsys2022}、CID~\cite{cid2023}、IDGenRec~\cite{idgenrec2024sigir}、GenRec~\cite{genrec2023} 与 GNPR-SID~\cite{gnprsid2025},以支撑“构造-对齐-生成”三层方法论与本文任务之间的对应关系。

\section{任务定义与符号约定}
设用户集合为 $\mathcal{U}=\{u_1,u_2,\ldots,u_M\}$,POI集合为 $\mathcal{P}=\{\ell_1,\ell_2,\ldots,\ell_N\}$。用户 $u$ 的第 $i$ 条签到定义为:
\begin{equation}
x_i=(u,\ell_i,t_i,g_i,c_i),
\label{eq:c2-checkin}
\end{equation}
式中:\symline{$u$}{用户标识;\\}
\hphantom{式中:}\symline{$\ell_i$}{第 $i$ 次签到对应的POI标识;\\}
\hphantom{式中:}\symline{$t_i$}{签到时间戳;\\}
\hphantom{式中:}\symline{$g_i=(lat_i,lon_i)$}{经纬度坐标;\\}
\hphantom{式中:}\symline{$c_i$}{POI类别语义标签。}

用户轨迹表示为:
\begin{equation}
\mathcal{T}_u=\{x_1,x_2,\ldots,x_n\},\quad t_1<t_2<\cdots<t_n.
\label{eq:c2-traj}
\end{equation}
式中:\symline{$\mathcal{T}_u$}{用户 $u$ 的时间有序签到序列;\\}
\hphantom{式中:}\symline{$n$}{轨迹长度;\\}
\hphantom{式中:}\symline{$t_1<t_2<\cdots<t_n$}{轨迹按时间严格递增排序。}

Next POI任务可表示为学习映射:
\begin{equation}
f:\mathcal{T}_u\mapsto \hat{\ell}_{n+1},\quad \hat{\ell}_{n+1}\in\mathcal{P},
\label{eq:c2-task}
\end{equation}
式中:\symline{$f$}{预测函数;\\}
\hphantom{式中:}\symline{$\hat{\ell}_{n+1}$}{模型预测的下一POI;\\}
\hphantom{式中:}\symline{$\mathcal{P}$}{候选POI全集。}
使真实下一POI $\ell_{n+1}$ 在候选排序中尽可能靠前。若输出Top-$K$列表,记为 $\hat{\mathbf{y}}_u=[\hat{\ell}^{(1)},\ldots,\hat{\ell}^{(K)}]$。

为减少后续章节符号歧义,表\ref{tab:c2-notation}给出本文高频符号及含义。
\begin{table}[htbp]
    \cTwoTblStyle
    \caption{核心符号说明}
    \label{tab:c2-notation}
    \begin{tabular}{p{0.30\textwidth}p{0.62\textwidth}}
        \toprule
        符号 & 含义 \\
        \midrule
        $\mathcal{U},\mathcal{P}$ & 用户集合与POI集合 \\
        $\mathcal{T}_u$ & 用户 $u$ 的时间有序签到轨迹 \\
        $x_i=(u,\ell_i,t_i,g_i,c_i)$ & 第 $i$ 次签到记录(用户、POI、时间、坐标、类别) \\
        $T_z$ & 第 $z$ 个时间槽(Time Slot) \\
        $\boldsymbol{\xi}^{out},\boldsymbol{\xi}^{in}$ & 小模型中POI转出/转入表示 \\
        $\mathbf{E}_{gps},\mathbf{E}_{poi}$ & 大模型侧地理编码与POI对齐表示 \\
        $\hat{\mathbf{y}}_u$ & 模型输出的Top-$K$推荐列表 \\
        $r_n$ & 测试样本中真实POI的排序位次 \\
        $\mathcal{L}_{\text{small}},\mathcal{L}_{\text{align}},\mathcal{L}_{\text{llm}}$ & 小模型损失、跨空间对齐损失与大模型任务损失 \\
        \bottomrule
    \end{tabular}
\end{table}

\subsection{任务边界与建模假设}
为保证论证聚焦,本文采用如下假设与边界设置:
\begin{enumerate}
    \item \textbf{时间先验可用}:训练与测试数据均包含可解析时间戳,且可划分为统一时段;
    \item \textbf{地理信息可用}:POI具备经纬度坐标,允许构建地理约束与距离相关特征;
    \item \textbf{用户历史可观测}:每个测试样本至少存在最小历史长度,以支持序列建模;
    \item \textbf{离线评估优先}:本文主实验聚焦离线Top-$K$评估,不直接讨论在线A/B系统收益。
\end{enumerate}
这些边界并不削弱问题价值,而是用于保证比较公平与可复现。对超出边界的场景(如极短轨迹、无坐标、实时概念漂移),将在第\ref{chap:conclusion}章与附录中讨论可能扩展路径。

\subsection{评价指标与实验协议说明}
本文采用 Acc@K、MRR 与 NDCG@K 评价排序质量,对应公式定义为:
\begin{equation}
\text{Acc@K}=\frac{1}{|\mathcal{D}|}\sum_{n=1}^{|\mathcal{D}|}\mathbf{1}(r_n\le K),
\label{eq:c2-acc}
\end{equation}
式中:\symline{$\mathcal{D}$}{测试样本集合;\\}
\hphantom{式中:}\symline{$|\mathcal{D}|$}{测试样本总数;\\}
\hphantom{式中:}\symline{$r_n$}{第 $n$ 个样本中真实POI的排序位次;\\}
\hphantom{式中:}\symline{$\mathbf{1}(\cdot)$}{指示函数,条件满足时取1,否则取0。}

\begin{equation}
\text{MRR}=\frac{1}{|\mathcal{D}|}\sum_{n=1}^{|\mathcal{D}|}\frac{1}{r_n},
\label{eq:c2-mrr}
\end{equation}
式中:\symline{$r_n$}{越小表示真实POI排名越靠前;\\}
\hphantom{式中:}\symline{$\frac{1}{r_n}$}{第 $n$ 个样本的倒数排名得分;\\}
MRR表示全体样本倒数排名的平均值。

\begin{equation}
\text{NDCG@K}=\frac{1}{|\mathcal{D}|}\sum_{n=1}^{|\mathcal{D}|}\frac{\mathbf{1}(r_n\le K)}{\log_2(r_n+1)}.
\label{eq:c2-ndcg}
\end{equation}
式中:\symline{$\log_2(r_n+1)$}{位置折损项;\\}
\hphantom{式中:}\symline{$\mathbf{1}(r_n\le K)$}{真实POI是否进入Top-$K$;\\}
NDCG@K对前排命中赋予更高权重。Acc@K反映命中能力,MRR强调首个正确结果位置,NDCG@K更关注头部排序质量。

本节仅说明指标含义与使用理由。数据划分、负采样、显著性检验和实现细节统一在第\ref{chap:exp}章给出。

\subsubsection{指标解释补充}
在Next POI场景中,不同指标对应不同系统目标:
\begin{enumerate}
    \item \textbf{Acc@1} 反映“第一推荐是否可直接点击”,通常对用户体验最敏感;
    \item \textbf{Acc@5/10} 反映候选列表覆盖能力,适用于存在二次筛选交互的场景;
    \item \textbf{MRR} 反映正确答案的平均前移程度,能够区分“命中但排名靠后”与“命中且靠前”;
    \item \textbf{NDCG@K} 对头部位置赋予更高权重,适合评估排序质量而非仅命中与否。
\end{enumerate}
因此,本文不以单一指标给出结论,而是从“首位可用性、候选覆盖、排序质量”三个维度综合判断方法有效性。

\section{研究空白与本文建模原则}
基于上述综述,本文将研究空白归纳为以下三点:
\begin{enumerate}
    \item \textbf{Gap-1:} 小模型强结构、弱语义,难以覆盖复杂意图表达与跨场景泛化;
    \item \textbf{Gap-2:} 大模型强语义、弱空间,难以稳定保持地理连续性与可达性约束;
    \item \textbf{Gap-3:} 缺少面向工程部署的统一协同训练方案,难以兼顾效果与成本。
\end{enumerate}

据此提出本文的建模原则:
\begin{enumerate}
    \item \textbf{DP-1(结构先验显式化):} 在小模型侧显式建模时间异质性与双向转移关系;
    \item \textbf{DP-2(空间语义对齐):} 在大模型侧引入地理编码与POI结构对齐模块;
    \item \textbf{DP-3(协同训练可部署):} 通过两阶段训练和参数高效微调实现稳定融合。
\end{enumerate}

相应地,本文联合优化目标写为:
\begin{equation}
\mathcal{L}=\mathcal{L}_{\text{small}}+\lambda_1\mathcal{L}_{\text{align}}+\lambda_2\mathcal{L}_{\text{llm}}+\lambda_3\mathcal{L}_{\text{reg}},
\label{eq:c2-joint}
\end{equation}
式中:\symline{$\mathcal{L}_{\text{small}}$}{小模型时空结构学习损失;\\}
\hphantom{式中:}\symline{$\mathcal{L}_{\text{align}}$}{跨空间对齐损失;\\}
\hphantom{式中:}\symline{$\mathcal{L}_{\text{llm}}$}{大模型任务损失;\\}
\hphantom{式中:}\symline{$\mathcal{L}_{\text{reg}}$}{正则化项;\\}
\hphantom{式中:}\symline{$\lambda_1,\lambda_2,\lambda_3$}{各损失项的权重系数。}

\subsection{从研究空白到研究问题(RQ)映射}
为保证后续实验可以直接检验建模假设,本文将 Gap/DP 与 RQ 的映射关系明确如下:
\begin{enumerate}
    \item Gap-1 对应 DP-1,并在 RQ2 中通过小模型消融验证;
    \item Gap-2 对应 DP-2,并在 RQ3 中通过GCIM/PAM诊断验证;
    \item Gap-3 对应 DP-3,并在 RQ4、RQ5 中通过协同收益与效率评估验证。
\end{enumerate}
该映射关系使论文论证链条从“问题识别”到“方法设计”再到“证据验证”保持一致,避免章节之间出现目标漂移。

\section{本章小结}
本章围绕“综述评估、定义统一、空白凝练”三条主线完成了问题分析。首先,系统比较了小模型与大模型在结构建模、语义泛化与工程开销上的能力边界;其次,给出了全文唯一任务定义与符号约定,避免后续章节重复描述;最后,将文献中的共性不足归纳为三类研究空白,并据此提出结构先验显式化、空间语义对齐和协同训练可部署三项建模原则。下一章将按照这些原则展开具体模型设计与训练机制说明。
