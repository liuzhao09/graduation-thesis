\chapter{相关研究与问题分析}
\label{chap:related}

\section{下一兴趣点推荐任务概述}
下一兴趣点推荐(Next Point-of-Interest Recommendation, Next POI)任务通常定义为:给定用户历史签到序列,预测其下一时刻最可能访问的POI。该任务本质上同时依赖三类信息:
\begin{enumerate}
    \item 序列依赖:近期访问行为对下一跳具有强约束;
    \item 空间依赖:地理邻近性与城市功能分区影响候选分布;
    \item 语义依赖:POI类别、时间上下文与用户意图决定可达性与偏好。
\end{enumerate}

经典工作将该问题视为序列建模任务,Where You Like to Go Next~\cite{cheng2013where} 和 Latent Behavior Patterns~\cite{he2016inferring} 具有代表性。后续研究逐渐扩展到时空序列深度网络与图结构建模框架,例如 HST-LSTM~\cite{HST-LSTM}、LSTPM~\cite{LSTPM}、STAN~\cite{STAN}、GETNext~\cite{GETNext}、STHGCN~\cite{STHGCN} 和 GraphFlashback~\cite{GraphFlashback}。此外,随着城市计算与地理建模发展,学界开始关注多尺度区域结构、跨模态语义特征与地理知识融合,代表工作包括 TSPN-RA~\cite{TSPN-RA}、M3PA~\cite{M3PA} 和 City Foundation Models~\cite{Pasquale2024City}。

\section{推荐系统基础与评价指标}
\subsection{从静态推荐到序列推荐}
传统推荐系统通常学习用户 $u$ 与项目 $i$ 的偏好函数 $\hat{y}_{ui}=f(u,i)$。在POI场景中,项目集合对应地点集合 $\mathcal{L}$,而用户状态还受历史轨迹影响,因此预测函数应写为
\begin{equation}
\hat{y}_{u,t,i}=f(u,\mathcal{T}_{u,1:t-1},i),
\label{eq:c2-score}
\end{equation}
如式(\ref{eq:c2-score})所示,其中 $\mathcal{T}_{u,1:t-1}$ 表示截至时刻 $t-1$ 的签到序列。与静态推荐相比,Next POI 推荐更强调条件分布 $p(\ell_t\mid \ell_{1:t-1},u,\text{context})$ 的建模。

\subsection{典型优化目标}
点式学习(point-wise)常以交叉熵最小化为目标:
\begin{equation}
\mathcal{L}_{\text{CE}}=-\sum_{(u,t)}\log p(\ell_t\mid \ell_{1:t-1},u).
\label{eq:c2-ce}
\end{equation}
式(\ref{eq:c2-ce})对应概率分类目标。对式学习(pair-wise)常采用 BPR~\cite{FPMC}:
\begin{equation}
\mathcal{L}_{\text{BPR}}=-\sum_{(u,i,j)}\log \sigma(\hat{y}_{u,i}-\hat{y}_{u,j})+\lambda\|\Theta\|_2^2,
\label{eq:c2-bpr}
\end{equation}
见式(\ref{eq:c2-bpr}),其中 $i$ 为真实下一POI,$j$ 为负采样POI。PRME~\cite{PRME} 进一步将排序学习与度量空间结合,在新地点推荐中具有较强表现。

\subsection{常用评价指标}
为衡量排序质量,本文后续采用 Accuracy@K、MRR 和 NDCG@K:
\begin{equation}
\text{Acc@K}=\frac{1}{|\mathcal{D}|}\sum_{n=1}^{|\mathcal{D}|}\mathbf{1}(r_n\le K),
\label{eq:c2-acc}
\end{equation}
\begin{equation}
\text{MRR}=\frac{1}{|\mathcal{D}|}\sum_{n=1}^{|\mathcal{D}|}\frac{1}{r_n},
\label{eq:c2-mrr}
\end{equation}
\begin{equation}
\text{NDCG@K}=\frac{1}{|\mathcal{D}|}\sum_{n=1}^{|\mathcal{D}|}\frac{\mathbf{1}(r_n\le K)}{\log_2(r_n+1)},
\label{eq:c2-ndcg}
\end{equation}
三项指标分别由式(\ref{eq:c2-acc})、式(\ref{eq:c2-mrr})和式(\ref{eq:c2-ndcg})定义,其中 $r_n$ 为第 $n$ 个样本中真实POI的排名。Acc@K反映命中能力,MRR强调首个正确结果的位置,NDCG@K则更关注头部排序质量。

\section{小模型路线:序列与图方法}
\subsection{序列建模方法}
早期方法以马尔可夫链与矩阵分解为主,代表模型 FPMC~\cite{FPMC} 与 PRME~\cite{PRME} 强调“用户偏好+转移模式”的联合建模。深度学习阶段中,RNN/LSTM成为主流,ST-RNN~\cite{ST-RNN} 显式引入时空上下文,HST-LSTM~\cite{HST-LSTM} 在层级结构中建模短期与长期偏好,LSTPM~\cite{LSTPM} 通过非局部操作增强长依赖建模,STAN~\cite{STAN} 进一步结合时空注意力提高表达能力。近年来,针对长期兴趣漂移、频率偏置与行为稀疏问题,CLSPRec~\cite{CLSPRec} 与 FHCRec~\cite{FHCRec} 采用对比学习强化表征鲁棒性。

在数学形式上,序列方法一般遵循以下状态更新:
\begin{equation}
\mathbf{h}_t=\Phi(\mathbf{x}_t,\mathbf{h}_{t-1};\Theta),
\label{eq:c2-rnn}
\end{equation}
其形式见式(\ref{eq:c2-rnn}),其中 $\mathbf{x}_t$ 是当前POI及上下文嵌入,$\Phi$ 可实例化为RNN或LSTM单元。若引入注意力机制,常写为
\begin{equation}
\alpha_{t,k}=\frac{\exp(\mathbf{q}_t^\top \mathbf{k}_k)}{\sum_{j=1}^{t-1}\exp(\mathbf{q}_t^\top \mathbf{k}_j)},\quad
\mathbf{c}_t=\sum_{k=1}^{t-1}\alpha_{t,k}\mathbf{v}_k.
\label{eq:c2-attn}
\end{equation}
如式(\ref{eq:c2-attn})所示。最终预测分布通常由
\begin{equation}
p(\ell_{t+1}\mid \mathcal{T}_{u,1:t})=\text{Softmax}(\mathbf{W}_o[\mathbf{h}_t\|\mathbf{c}_t]+\mathbf{b}_o)
\label{eq:c2-softmax}
\end{equation}
给出,对应式(\ref{eq:c2-softmax})。该范式能够刻画短期转移,但当轨迹跨度增长、空间约束增强时,单序列编码难以充分表达跨用户共享结构。

序列方法在中短序列预测上表现稳定,但普遍存在以下不足:
\begin{enumerate}
    \item 在稀疏轨迹和冷启动场景下,参数学习依赖充足监督信号;
    \item 对“时间段差异”与“转入/转出双向偏好”刻画不充分;
    \item 对跨用户高阶关系利用有限,难以吸收群体迁移知识。
\end{enumerate}

\subsection{图建模方法}
图神经网络通过构建POI转移图、用户-POI二部图、超图或动态图显式学习高阶关系。代表方法包括 GETNext~\cite{GETNext}、GraphFlashback~\cite{GraphFlashback}、STHGCN~\cite{STHGCN}、SNPM~\cite{SNPM}、AGCL~\cite{AGCL} 与 DCPR~\cite{DCPR}。这类方法可跨用户传播结构信息,在稀疏场景中显著提升表示质量。为增强时间建模能力,ROTAN~\cite{ROTAN} 引入旋转时序注意力,MTNet~\cite{MTNet} 采用 mobility tree 学习时间槽偏好。

图模型的核心可抽象为消息传递:
\begin{equation}
\mathbf{h}_i^{(l+1)}=\sigma\!\left(\mathbf{W}_0^{(l)}\mathbf{h}_i^{(l)}+\sum_{j\in\mathcal{N}(i)}\beta_{ij}^{(l)}\mathbf{W}_1^{(l)}\mathbf{h}_j^{(l)}\right),
\label{eq:c2-msg}
\end{equation}
其形式见式(\ref{eq:c2-msg}),其中 $\beta_{ij}^{(l)}$ 为归一化邻接权重或注意力权重。若考虑时空边特征 $\mathbf{e}_{ij}=[\Delta t_{ij},d_{ij}]$,可进一步写为
\begin{equation}
\beta_{ij}^{(l)}\propto \exp\!\big(\text{MLP}([\mathbf{h}_i^{(l)}\|\mathbf{h}_j^{(l)}\|\mathbf{e}_{ij}])\big).
\label{eq:c2-edge}
\end{equation}
如式(\ref{eq:c2-edge})所示。这使模型同时编码“谁与谁相连”与“何时、隔多远发生转移”。

在POI任务中,动态图通常记为 $\mathcal{G}_t=(\mathcal{V},\mathcal{E}_t)$,并基于滑动窗口更新边集:
\begin{equation}
\mathcal{E}_t=\{(\ell_{k},\ell_{k+1})\mid t-k\leq w\}.
\label{eq:c2-dyn}
\end{equation}
式(\ref{eq:c2-dyn})刻画了局部时间窗内的转移更新规则。该建模可反映近期行为变化,但随时间推移会带来图更新成本上升、训练不稳定和邻居爆炸问题。

图方法尽管性能强,但仍面临以下挑战:
\begin{enumerate}
    \item 图结构通常静态,难以反映不同时间段转移动态;
    \item 多源异构信息(文本语义、类别语义、空间连续坐标)融合成本高。
\end{enumerate}

\section{大模型路线:LLM驱动推荐}
\subsection{LLM在推荐系统中的应用}
在通用推荐领域,研究者已探索将LLM与协同信号、检索机制和图结构结合。CoLLM~\cite{CoLLM} 将协同嵌入注入语言模型语义空间;CoRAL~\cite{CoRAL} 通过检索增强改善长尾推荐;LLMRec~\cite{LLMRec} 采用图增强提升结构感知;ReLLa~\cite{ReLLa} 面向长历史行为引入检索式记忆。面向序列推荐,SeCor~\cite{SeCor} 与 LLaRA~\cite{LLaRA} 分别从语义-协同对齐和助理式推理角度验证了LLM在行为建模中的潜力。

若将推荐任务表述为条件生成,LLM目标可写为
\begin{equation}
\max_{\Theta}\sum_{(x,y)}\log p_{\Theta}(y\mid x),
\label{eq:c2-llm}
\end{equation}
见式(\ref{eq:c2-llm}),其中 $x$ 是由用户历史、候选集合和外部上下文构成的提示词,$y$ 为下一POI或其标识。参数高效微调常采用 LoRA 形式
\begin{equation}
\Delta \mathbf{W}=\mathbf{B}\mathbf{A},\quad \mathbf{W}'=\mathbf{W}+\Delta\mathbf{W},
\label{eq:c2-lora}
\end{equation}
对应式(\ref{eq:c2-lora}),在保持主干参数冻结的前提下降低训练成本。

\subsection{LLM在Next POI任务中的进展}
在Next POI方向,LLM4POI~\cite{LLM4POI} 通过轨迹提示将任务转化为生成问题,展现了冷启动潜力。GraphGPT~\cite{GraphGPT} 与 E4SRec~\cite{E4SRec} 虽非专门针对POI场景,但其“结构指令化”和“高效序列适配”思路对POI任务有重要启发。GA-LLM~\cite{liu2026gallm} 进一步指出文本LLM存在空间幻觉、长距离误判与目标POI缺失时泛化不足问题,并提出地理坐标注入模块与POI对齐模块缓解缺陷。

针对空间信息编码,一个关键思想是将连续坐标 $g=(lat,lon)$ 映射到离散地理单元 $q=\text{Quadkey}(g,z)$,再学习嵌入
\begin{equation}
\mathbf{e}_g=\text{Emb}(q)+\text{PE}_{\text{geo}}(g).
\label{eq:c2-geoemb}
\end{equation}
对应式(\ref{eq:c2-geoemb})。其中 $\text{PE}_{\text{geo}}$ 可采用可学习傅里叶特征,形式为
\begin{equation}
\text{PE}_{\text{geo}}(g)=\big[\sin(2\pi \mathbf{B}g),\cos(2\pi \mathbf{B}g)\big].
\label{eq:c2-fourier}
\end{equation}
如式(\ref{eq:c2-fourier})所示。该方案能缓解高精度坐标分词导致的序列膨胀问题,并增强空间邻近可分性。

\subsection{现阶段主要瓶颈}
综合现有LLM-POI研究,仍有三类关键瓶颈:
\begin{enumerate}
    \item \textbf{坐标稀疏语义问题}:经纬度被分词后难以保持几何邻近关系;
    \item \textbf{转移先验缺失问题}:纯文本LLM难以隐式恢复POI图转移规律;
    \item \textbf{空间幻觉问题}:可能生成地理上不合理或超出合理活动范围的候选。
\end{enumerate}

\section{研究空白与本文问题建模}
综合上述文献,当前研究的关键矛盾可归纳为:
\begin{enumerate}
    \item 小模型时空结构建模强,但语义泛化能力不足;
    \item 大模型语义理解强,但地理连续性与转移结构建模弱。
\end{enumerate}

由此产生本文的核心研究问题:如何在统一框架中同时保持小模型对时空转移的精细刻画能力,以及大模型对异构语义的泛化推理能力,并在实际可部署约束下实现性能与效率平衡。

为解决该问题,本文采用“\textbf{小模型增强时空结构 + 大模型增强语义推理 + 融合对齐训练}”的技术路线:
\begin{enumerate}
    \item 在小模型侧构建时间增强序列动态图与双向转移建模,学习高质量时空表示;
    \item 在大模型侧引入地理坐标注入模块(GCIM)与POI对齐模块(PAM),增强空间一致性与转移先验;
    \item 通过嵌入对齐与两阶段训练,将小模型知识稳定注入LLM,实现协同优化。
\end{enumerate}

上述融合框架可形式化为联合优化问题:
\begin{equation}
\mathcal{L}=\mathcal{L}_{\text{small}}+\lambda_1\mathcal{L}_{\text{align}}+\lambda_2\mathcal{L}_{\text{llm}}+\lambda_3\mathcal{L}_{\text{reg}},
\label{eq:c2-joint}
\end{equation}
即式(\ref{eq:c2-joint}),其中 $\mathcal{L}_{\text{small}}$ 学习时空结构先验,$\mathcal{L}_{\text{align}}$ 约束小模型与LLM语义空间对齐,$\mathcal{L}_{\text{llm}}$ 对应生成或分类目标,$\mathcal{L}_{\text{reg}}$ 控制参数复杂度。该路线既保留了小模型对局部时空依赖的刻画能力,又利用了LLM的上下文语义推理优势,并通过结构化注入与参数高效微调降低工程成本,更适合真实出行场景下的可部署需求。

\section{本章小结}
本章围绕“方法演进脉络与现实瓶颈”对Next POI研究进行了系统梳理。整体来看,FPMC~\cite{FPMC}、PRME~\cite{PRME} 所代表的早期范式在可解释性和训练稳定性方面具有优势,但对复杂时空上下文的表达能力有限;ST-RNN~\cite{ST-RNN}、HST-LSTM~\cite{HST-LSTM}、STAN~\cite{STAN} 所代表的序列深度模型显著提升了时序建模能力,但对跨用户高阶迁移知识利用不足;GETNext~\cite{GETNext}、SNPM~\cite{SNPM}、GraphFlashback~\cite{GraphFlashback} 所代表的图方法在结构关系学习上更具优势,但在动态更新和异构融合方面存在工程复杂度偏高的问题。

在大模型方向,LLM4POI~\cite{LLM4POI}、LLaRA~\cite{LLaRA}、SeCor~\cite{SeCor}、CoLLM~\cite{CoLLM} 展现了语义泛化与冷启动潜力,但空间连续性建模和转移先验注入仍是核心短板。GA-LLM~\cite{liu2026gallm} 提供了具有针对性的改进思路,说明“地理编码增强+结构先验对齐”是提升LLM在时空推荐任务中可靠性的关键路径。

基于以上分析,本文后续章节将以“结构先验与语义推理协同优化”为主线,分别从小模型分支的时间增强动态图建模、大模型分支的地理坐标注入与POI对齐机制、以及端到端协同训练策略三个维度展开,目标是在准确率、鲁棒性与部署效率之间实现更优平衡。
