\chapter*{结论与展望}
\addcontentsline{toc}{chapter}{结论与展望}
\label{chap:conclusion}

\section*{结论}
本文围绕Next POI任务中的“结构建模与语义推理协同”问题,提出了大小模型协同学习框架。研究结论可概括为三点:
\begin{enumerate}
    \item 在小模型侧,时间增强序列动态图与双向转移机制提升了对时段异质行为和复杂路径转移的刻画能力;
    \item 在大模型侧,GCIM与PAM缓解了空间幻觉与转移先验缺失问题,增强了冷启动与跨城场景下的泛化表现;
    \item 在统一协同训练下,模型在准确率、鲁棒性与部署开销之间实现了更优平衡,验证了“结构先验 + 语义推理”融合路线的有效性。
\end{enumerate}

\section*{展望}
后续可进一步从以下方向展开:
\begin{enumerate}
    \item 引入更丰富的多模态上下文(如地理文本、图像或交通信号)以增强场景理解能力;
    \item 研究更细粒度的跨城迁移与在线自适应更新机制,提升动态环境下的持续学习能力;
    \item 在工业级部署中联合优化召回、重排与生成模块,进一步降低端到端时延与成本。
\end{enumerate}
