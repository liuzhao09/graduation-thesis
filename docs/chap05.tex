% 结论章按学位论文规范单独成章,不设置章序号
\chapter*{结论}
\addcontentsline{toc}{chapter}{结论}
\markboth{结论}{}
\label{chap:conclusion}

本文围绕Next POI任务中的“结构建模与语义推理协同”问题展开研究,形成了面向小模型与大模型协同优化的完整技术路线。结合前文理论分析与实验结果,本文结论可归纳为以下三点。
\begin{enumerate}
    \item 提出了大小模型协同学习框架,建立了“时空结构先验建模 + 语义推理增强 + 跨空间对齐训练”的统一方法范式,使Next POI任务在同一框架内兼顾结构表达能力与语义泛化能力;
    \item 建立了以时间增强序列动态图和双向转移建模为核心的小模型分支,提升了模型对时段异质行为和复杂路径转移的刻画能力,并在主流序列与图基线对比中取得稳定性能增益;
    \item 提出了GCIM与PAM两类大模型增强机制,缓解了地理坐标语义错位与转移先验缺失问题,在空间一致性、跨场景泛化与效率开销之间取得了更优平衡,验证了协同路线的有效性与可部署性。
\end{enumerate}

\section*{主要研究结论}
围绕全文提出的RQ1--RQ5,本文可进一步总结如下:
\begin{enumerate}
    \item {\heiti\addCJKfontfeatures{AutoFakeBold=2}关于总体有效性(RQ1):}协同框架在主流评估指标上保持一致改进,说明“结构增强 + 语义增强”能够形成稳定增益,而非局部指标优化;
    \item {\heiti\addCJKfontfeatures{AutoFakeBold=2}关于小模型机制(RQ2):}时间增强序列动态图、双向转移与动态图权重均具有独立贡献,证明小模型分支对时空规律的建模不是单模块偶然收益;
    \item {\heiti\addCJKfontfeatures{AutoFakeBold=2}关于大模型机制(RQ3):}GCIM有效缓解空间错位,PAM有效增强转移先验注入,二者联合时在冷启动与跨城场景更稳健;
    \item {\heiti\addCJKfontfeatures{AutoFakeBold=2}关于协同机理(RQ4):}协同模型在远跳错误与语义误判上有针对性改善,体现出结构约束对语义偏差的校正作用;
    \item {\heiti\addCJKfontfeatures{AutoFakeBold=2}关于效率部署(RQ5):}借助LoRA与模块化设计,模型在保持性能增益的同时控制了训练与推理开销,具备工程落地可行性。
\end{enumerate}

\section*{创新点归纳}
从方法与实证两方面看,本文创新性体现在以下三个维度:
\begin{enumerate}
    \item {\heiti\addCJKfontfeatures{AutoFakeBold=2}问题层创新:}将Next POI任务中的“结构信息难注入LLM”问题显式化,并以协同框架而非单模型增强作为核心解法;
    \item {\heiti\addCJKfontfeatures{AutoFakeBold=2}方法层创新:}提出GCIM与PAM两条互补注入路径,实现地理连续性与POI先验在语义空间中的可学习表达;
    \item {\heiti\addCJKfontfeatures{AutoFakeBold=2}验证层创新:}构建按RQ组织的证据闭环,覆盖主结果、消融、诊断、效率与威胁分析,提升结论可信度与可解释性。
\end{enumerate}

\section*{研究不足}
尽管本文取得了阶段性结果,仍存在以下不足:
\begin{enumerate}
    \item 目前主要基于公开签到数据开展离线评估,对线上交互反馈与实时决策约束考虑有限;
    \item 跨域泛化验证仍以城市级迁移为主,对更强分布漂移(跨国家、跨文化、跨业态)评估尚不充分;
    \item 外部上下文(天气、交通事件、节假日)尚未系统纳入,极端场景下仍存在提升空间。
\end{enumerate}

在研究展望方面,后续工作可从以下方向继续推进:
\begin{enumerate}
    \item 引入更丰富的多模态上下文信息(如地理文本、图像与交通信号),进一步增强模型对复杂城市场景的语义理解能力;
    \item 研究细粒度跨城迁移与在线自适应更新机制,提升模型在动态环境中的持续学习能力与长期稳定性;
    \item 在实际系统中持续优化生成链路与缓存更新机制,进一步降低端到端时延与计算成本,提升工程落地效率。
\end{enumerate}

\section*{结语}
总体而言,本文围绕“Next POI推荐中的结构-语义协同”给出了较完整的理论分析、方法设计与实验验证。相关结论表明,在时空推荐这一高约束任务中,单一路线难以同时兼顾精度、泛化与可解释性;通过结构先验与语义推理的协同学习,可获得更稳定且更具工程价值的预测能力。上述工作为后续研究提供了可复用的技术框架与分析基线。

% 结论结束后恢复普通页样式
\pagestyle{thesisnormal}
